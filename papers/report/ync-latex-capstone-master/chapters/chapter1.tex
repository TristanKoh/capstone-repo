% Chapter 1

\chapter{Introduction} % Main chapter title

\label{chapter1} % For referencing the chapter elsewhere, use \ref{Chapter1}

%----------------------------------------------------------------------------------------
\section{Motivation and significance}
Natural language forms the bread and butter of the legal industry, as expressed in contracts, judgements and legislation. There has been an increasing trend within the legal industry to adopt more machine learning techniques to automate and assist low level legal analysis. Within the specific context of data privacy, it would be useful to have a tool that assesses the possible data privacy risks of user policies. Such a tool would naturally use NLP techniques. Nevertheless, these tools should still be accessible to the layperson lawyer that might not be trained in data science.

While NLP techniques have substantially increased in performance in recent years, it has come at the cost of explainability of predictions because of the usage of neural networks which are architecturally more complex than traditional machine learning models. This lack of explainability of neural networks could potentially be a significant hindrance towards their adoption within the legal industry because the lawyer / law firm which uses these models still ultimately bear the responsibility of ensuring that the analysis is legally sound. 

However, the intersection in skillset between data science and legal analysis is still nascent and it is unrealistic to expect all legally trained personnel to be trained in data science to the extent required to interpret the predictions of machine learning models without aid.

Recent research within the legal NLP space have focused on building higher performing models and Text representation, but there have been comparatively few papers that assess the explainability of such models. Therefore, my capstone aims to bridge the gap between the lawyer and the data scientist by using Explainable AI techniques to explain the predictions of machine learning models in the context of predicting data privacy practices of app policies.

\subsection{Development and importance of explainable AI in legal technology}
In this chapter, I will demonstrate a few interesting features of LaTeX, and more importantly, provide examples of Figures, Tables, Equations and Code Snippets.
These may be easy to deal with on Word, but here it is another story.
The general idea is to keep these examples, copy-paste them, then modify them to fit your needs.
If you are seeing this content while reading \texttt{main.pdf}, please load the overall LaTeX project in your LaTeX editor, and open the \texttt{chapters/chaper1.tex}.
\\
\\
Done?
Ok let us move on then!
So by now, you should have noticed a few things:
\begin{itemize}
  \item Each sentence of the text is on a distinct line. Yet, sentences are still within the same paragraph.
  \item Backslash is an escape character, used at the beginning of LaTeX commands.
  \item To end a paragraph (or actually insert a newline), we can use the \textbackslash{}\textbackslash{} command.
\end{itemize}
Also now you know how to do a bullet list.

\subsection{Development and importance of data privacy regulation}
Chapter contain sections (defined with \textbackslash{}section\{Name of your section\}), subsections (\textbackslash{}subsection).

\section{Explanation of the APP-350 Corpus}
The APP-350 Corpus consists of 350 annotated Android app privacy policies. Each annotation consists of a practice and a modality.

A "privacy practice" (or "practice") describes a certain behaviour of an app that can have privacy implications (e.g., collection of a phone's device identifier or sharing of its location with ad networks). There are two modalities: PERFORMED (i.e. a practice is explicitly described as being performed) and NOT\_PERFORMED (i.e. a practice is explicitly described as not being performed).

As not all practices had modalities, altogether, 57 different categories were annotated. The following is a table of the practices and their descriptions.

\begin{table}[]
	\resizebox{\textwidth}{!}{%
	\begin{tabular}{ll}
	\hline
	Data Type                 & Description                                                                                    \\ \hline
	Contact                   & The policy describes collection of unspecified contact data.                                   \\
	Contact\_Address\_Book    & The policy describes collection of contact data from a user's address book on the phone.       \\
	Contact\_City             & The policy describes collection of the user's city.                                            \\ 
	Contact\_E\_Mail\_Address & The policy describes collection of the user's e-mail.                                          \\
	Contact\_Password         & The policy describes collection of the user's password.                                        \\
	Contact\_Phone\_Number    & The policy describes collection of the user's phone number.                                    \\
	Contact\_Postal\_Address  & The policy describes collection of the user's postal address.                                  \\
	Contact\_ZIP              & The policy describes collection of the user's ZIP code.                                        \\
	Demographic               & The policy describes collection of the user's unspecified demographic data.                    \\
	Demographic\_Age          & The policy describes collection of the user's age (including birth date and age range).        \\
	Demographic\_Gender       & The policy describes collection of the user's gender.                                          \\
	Identifier                & The policy describes collection of the user's unspecified identifiers.                         \\
	Identifier\_Ad\_ID        & The policy describes collection of the user's ad ID (such as the Google Ad ID).                \\
	Identifier\_Cookie\_or\_similar\_Tech & The policy describes collection of the user's HTTP cookies, flash cookies, pixel tags, or similar identifiers. \\
	Identifier\_Device\_ID    & The policy describes collection of the user's device ID (such as the Android ID).              \\
	Identifier\_IMEI          & The policy describes collection of the user's IMEI (International Mobile Equipment Identity).  \\
	Identifier\_IMSI          & The policy describes collection of the user's IMSI (International Mobile Subscriber Identity). \\
	Identifier\_IP\_Address   & The policy describes collection of the user's IP address.                                      \\
	Identifier\_MAC           & The policy describes collection of the user's MAC address.                                     \\
	Identifier\_Mobile\_Carrier           & The policy describes collection of the user's mobile carrier name or other mobile carrier identifier.          \\
	Identifier\_SIM\_Serial   & The policy describes collection of the user's SIM serial number.                               \\
	Identifier\_SSID\_BSSID   & The policy describes collection of the user's SSID or BSSID.                                   \\
	Location                  & The policy describes collection of the user's unspecified location data.                       \\
	Location\_Bluetooth       & The policy describes collection of the user's Bluetooth location data.                         \\
	Location\_Cell\_Tower     & The policy describes collection of the user's cell tower location data.                        \\
	Location\_GPS             & The policy describes collection of the user's GPS location data.                               \\
	Location\_IP\_Address     & The policy describes collection of the user's IP location data.                                \\
	Location\_WiFi            & The policy describes collection of the user's WiFi location data.                              \\
	SSO                       & The policy describes receiving data from an unspecified single sign on service.                \\
	Facebook\_SSO             & The policy describes receiving data from the Facebook single sign on service.                 
	\end{tabular}%
	}
	\caption{List of annotated data privacy practices and their descriptions.}
	\end{table}


The APP-350 Corpus was used in a broader project to train machine learning models to conduct a privacy census of 1,035,853 Android apps. In that project, the researchers downloaded the data privacy practices of all apps from the Play Store with more than 350 million installs (which totalled 247 apps) and 103 randomly selected apps with 5 million installs. In total, the researchers collected the data privacy policies of 350 apps.

All 350 policies were annotated by one of the authors, a lawyer with experience in data privacy law. To ensure reliability of annotations, 2 other law students were hired to double annotate 10\% of the corpus. With a mean of Krippendorff's $\alpha = 0.78$\footnote{Krippendorff's $\alpha$ is a measure of agreement, with $\alpha > 0.8$ indicating good agreement, $0.67 <= \alpha <= 0.8$ indicating fair agreement, and $\alpha < 0.67$ indicating doubtful agreement.}, the agreement between the annotations exceeded previous similar research.

For more information about how the Corpus was annotated, see the paper “MAPS: Scaling Privacy Compliance Analysis to a Million Apps”, Section 3, Pg 69 to 70. 

\subsection{Rationale for utilising the APP-350 corpus}

Since the focus of this capstone is to assess the interpretability of XAI models specifically within a legal context, this dataset was chosen for the following reasons:

\begin{enumerate}
	\item APP-350 contains real-world data privacy practices as they were scraped from Google PlayStore apps. Thus training XAI models on such a dataset would provide a realistic insight into the extent of which AI models are explainable in the legal context.
	\item Legal tech companies are also using such datasets to train models as part of their contract / document review products. By using APP-350 to train XAI models, the results can be used as a (simple)\footnote{The datasets used in industry are usually much larger and the models used are more complicated. However, APP-350 would be sufficiently complicated to serve as a toy example at an undergraduate level.} proxy for the explainability of models that are currently used in the industry.
	\item APP-350 is a labelled dataset, allowing easy validation of results. If an unlabelled dataset was used, unsupervised training would have to be conducted. The performance of the models would likely be much lower because NLP models for specific vocabulary like law are still not as sophisticated as models trained on general vocabulary. Further, there are few law specific labelled datasets to begin with. 
	\item APP-350 is labelled on both the sentence and segment (i.e. paragraph) level. This provides more granular data for training the AI models. 
\end{enumerate}


\section{Problem statement}

\section{Main findings and roadmap}

\section{Font Formatting Commands}
Similarly to Word, LaTeX provides simple formatting, including \textbf{bold}, \textit{italic}, \underline{underlined} and \texttt{ugly stuff}.
However, no underline or strikethrough by default.
You can also change the size of the text, using {\tiny tiny}, {\small small}, {\large large}, {\huge huge}.
These last commands work within a specific scope.
The scope can be specified using \{ and \}, with the \{ placed before the \textbackslash{}size command.

\subsection{Special characters}
LaTeX uses 10 special characters. Each of these characters has a special meaning.
\begin{enumerate}
  \item Ampersand (\&) is used in tables as a cell delimiter.
  \item Percent sign (\%) is used for commenting a line.
  \item Dollar sign (\$) is used to switch back/from mathematical notation mode.
  \item Hash sign (\#) is used to create macros --- you definitely do not want to go more in depth here.
  \item Underscore (\_ or \textunderscore) is used to indicate a subscript in maths mode, if you use it in text mode ( without using backslash in front to ''escape'' it), your project will not compile anymore. \textbf{You may want to read that twice, and remember it.} It is in a LaTeX template, therefore it must be true.
  \item Curly brackets (\{ and \}) or bracets are used by LaTeX commands, as you likely already noticed.
  \item Tilde (\textasciitilde) can be used to create a non-breaking space (so that both words are on the same line).
  \item Caret/Circumflex/Hat (\textasciicircum) is used to indicate superscript (exponent) in maths mode.
  \item Backslash (\textbackslash) is used in front of every command. You cannot simply escape it to print it, as \textbackslash{}\textbackslash{} create a new line.
\end{enumerate}
If you happen to insert some of these symbols in your text without either escaping (when possible) or using the correct command, your project will likely not compile.
Thus, you may want to be extra careful about that problem.
Note: the underscore issue may also be encountered with bibliography.
So if \texttt{bibtex} displays an error, it may also come from an underscore somewhere in the abstract, DOI or URL field.


In that case, simply use \textdollar{} (by the way, note that using the dollar sign in your text switches to mathematical notation. To actually print a dollar sign use the \textbackslash{}textdollar command).
The equation above has a label, meaning you can refer to it. The numbering system uses the chapter number (in this case 1), then the equation position within the chapter (1 again).
Example: Equation~\ref{eq:eq1} is an example of an equation in LaTeX{}.
In case you would like to have an equation without numbering it? Easy!
\begin{equation*}
t = a \times log_{2}(\frac{D}{W} + 1) + b
\end{equation*}

The only difference? The \textasteriskcentered{}  symbol in the \textbackslash{}begin\{equation\textbf{\textasteriskcentered}\}.
This also works with Figures and Tables.

\section{Code Snippets}

\begin{lstlisting}
  int main (int argc, char ** argc)
  {
    printf("Hello world!\n");
    return 0;
  }
\end{lstlisting}

This template uses the \texttt{lstlisting} package, which not the best for code snippets.
However, it works without any problem, while other packages may have compatibility issues.
Feel free to try alternative solutions, the best one being \texttt{minted}.

\section{Figures}
Figures are a bit tricky with LaTeX {\tiny(not as much as tables though)}.
Let us see a simple example below:
\begin{figure}[!h]
  \centering
    \includegraphics[width=0.9\textwidth]{figures/future.png}
  \caption{When a YNC alumni tells you that back in their days, they did not have LaTeX template and would write their report in latin on a papyrus.}
  \label{fig:future}
\end{figure}
You can refer to it: Figure~\ref{fig:future}.
This is possible thanks to the \textbackslash{}label command.
The figure should also be shown on the \hyperref[lst:figs]{List of Figures} page (note this other way of referring to another part of the manuscript!).
A common practice is use the following naming convention:
\begin{itemize}
  \item A prefix, indicating the nature of the object labelled: \texttt{eq} for equations, \texttt{fig} for figures, \texttt{tab} for tables.
  \item A colon.
  \item A unique name (easy to remember) describing your figure. Example: exp1confmatrix would suggest that the figure shows a confusion matrix for your experiment 1.
\end{itemize}

A few other points: The \textbackslash{}caption and \textbackslash{}label can be put either before or after the \textbackslash{}includegraphics command.
When you create a Figure, you need to provide placement information for LaTeX. LaTeX will usually not locate the figures \emph{exactly} where you want them.
The most common specifiers are: \texttt{h} (here), \texttt{b} (bottom of the page) and \texttt{t} (top). The \texttt{!} specifier tries to force LaTeX to put the image exactly at the location you specified (with mixed success though).
For a longer list of specifiers, please refer to: \url{https://en.wikibooks.org/wiki/LaTeX/Floats,_Figures_and_Captions}.

\subsection{Figure Size}
The size of the figure can be determined by the first parameter of the \textbackslash{}includegraphics command.
In this example, we set the size to be $0.9 \times \texttt{textwidth}$, or 90\% of the size of a column.
We could have used an absolute value in cm, e.g. \texttt{width=19cm}.

\subsection{Supported Formats}
Use standard formats, such as PNG, PDF, JPG.
LaTeX also supports other formats, such as EPS.
\textbf{Rule of thumb: use PDF as much as you can, as it uses vector graphics, making it easy to scale the figure to very large format without problems.}

\subsection{Multiple images in one figure}
You can also create complex figures with multiple images.
Here is an example, which uses a $2\times2$ layout.
The overall figure can be referred as Figure~\ref{fig:drake}.
\begin{figure}[!h]
  \begin{subfigure}[t]{.5\textwidth}
    \centering
    \includegraphics[width=\linewidth]{figures/draketl.png}
    %\caption{We could totally insert a caption here}
    %\label{fig:draketl}
  \end{subfigure}
  \hfill
  \begin{subfigure}[t]{.5\textwidth}
    \centering
    \includegraphics[width=\linewidth]{figures/draketr.png}
    %\caption{We could totally insert a caption here}
        %\label{fig:draketr}
  \end{subfigure}

  %\medskip
  % the medskip will have white space between both lines
  \begin{subfigure}[t]{.5\textwidth}
    \centering
    \includegraphics[width=\linewidth]{figures/drakebl}
    %\caption{We could totally insert a caption here}
        %\label{fig:drakebl}
  \end{subfigure}
  \hfill
  \begin{subfigure}[t]{.5\textwidth}
    \centering
    \includegraphics[width=\linewidth]{figures/drakebr}
    %\caption{We could totally insert a caption here}
    %\label{fig:drakebr}
  \end{subfigure}
  \caption{Example of a complex figures on a $2\times2$ layout.}
  \label{fig:drake}
\end{figure}

\section{Tables}
Tables can be a nightmare in LaTeX.
The easiest way to deal with tables in LaTeX is to use some online tools.
My favorite so far: \url{https://www.tablesgenerator.com/}

Here is an example of confusion matrix generated:
\begin{table}[!h]
  \resizebox{\textwidth}{!}{
\begin{tabular}{|c|c|ccccccccc}
\cline{1-2}
Chest (C)                         & -     &                                                                          &                                                                           &                                                                          &                          &                            &                           &                           &                            &                                                                            \\ \cline{1-3}
Chest (ND)  & *     & \multicolumn{1}{c|}{-}                                                   &                                                                           &                                                                          &                          &                            &                           &                           &                            &                                                                            \\ \cline{1-4}
Chest (D)   & *     & \multicolumn{1}{c|}{-}                                                   & \multicolumn{1}{c|}{*}                                                    &                                                                          &                          &                            &                           &                           &                            &                                                                            \\ \cline{1-5}
Ear          & -     & \multicolumn{1}{c|}{-}                                                   & \multicolumn{1}{c|}{-}                                                    & \multicolumn{1}{c|}{}                                                    &                          &                            &                           &                           &                            &                                                                            \\ \cline{1-6}
Thigh       & *     & \multicolumn{1}{c|}{-}                                                   & \multicolumn{1}{c|}{*}                                                    & \multicolumn{1}{c|}{*}                                                   & \multicolumn{1}{c|}{-}   &                            &                           &                           &                            &                                                                            \\ \cline{1-7}
Neck        & -     & \multicolumn{1}{c|}{*}                                                   & \multicolumn{1}{c|}{-}                                                    & \multicolumn{1}{c|}{*}                                                   & \multicolumn{1}{c|}{-}   & \multicolumn{1}{c|}{-}     &                           &                           &                            &                                                                            \\ \cline{1-8}
Palm        & -     & \multicolumn{1}{c|}{}                                                    & \multicolumn{1}{c|}{*}                                                    & \multicolumn{1}{c|}{-}                                                   & \multicolumn{1}{c|}{-}   & \multicolumn{1}{c|}{-}     & \multicolumn{1}{c|}{-}    &                           &                            &                                                                            \\ \cline{1-9}
Thumb       & -     & \multicolumn{1}{c|}{*}                                                   & \multicolumn{1}{c|}{-}                                                    & \multicolumn{1}{c|}{*}                                                   & \multicolumn{1}{c|}{-}   & \multicolumn{1}{c|}{-}     & \multicolumn{1}{c|}{*}    & \multicolumn{1}{c|}{*}    &                            &                                                                            \\ \cline{1-10}
Inner Wrist & *     & \multicolumn{1}{c|}{-}                                                   & \multicolumn{1}{c|}{-}                                                    & \multicolumn{1}{c|}{-}                                                   & \multicolumn{1}{c|}{*}   & \multicolumn{1}{c|}{*}     & \multicolumn{1}{c|}{*}    & \multicolumn{1}{c|}{-}    & \multicolumn{1}{c|}{*}     &                                                                            \\ \hline
Outer Wrist & -     & \multicolumn{1}{c|}{*}                                                   & \multicolumn{1}{c|}{-}                                                    & \multicolumn{1}{c|}{*}                                                   & \multicolumn{1}{c|}{-}   & \multicolumn{1}{c|}{-}     & \multicolumn{1}{c|}{-}    & \multicolumn{1}{c|}{*}    & \multicolumn{1}{c|}{-}     & \multicolumn{1}{c|}{-}                                                     \\ \hline
  & Belly & \multicolumn{1}{c|}{\begin{tabular}[c]{@{}c@{}}Chest\\ (C)\end{tabular}} & \multicolumn{1}{c|}{\begin{tabular}[c]{@{}c@{}}Chest\\ (ND)\end{tabular}} & \multicolumn{1}{c|}{\begin{tabular}[c]{@{}c@{}}Chest\\ (D)\end{tabular}} & \multicolumn{1}{c|}{Ear} & \multicolumn{1}{c|}{Thigh} & \multicolumn{1}{c|}{Neck} & \multicolumn{1}{c|}{Palm} & \multicolumn{1}{c|}{Thumb} & \multicolumn{1}{c|}{\begin{tabular}[c]{@{}c@{}}Inner\\ Wrist\end{tabular}} \\ \hline
\end{tabular}
}
\caption{Post-hoc comparisons between body parts. - shows no significant difference ($p>.05$), \textasteriskcentered{} shows differences ($p<.05$).}
\label{tab:posthoc}
\end{table}

Note that a table is actually a container for another type of LaTeX object, \emph{tabular}.
Tables come with captions and label, allowing us to refer to Table~\ref{tab:posthoc}.
Another interesting point is that the \textbackslash{}begin\{tabular\} command uses characters.
These characters specify how the text should be centered within each cell: \texttt{c} means centered, \texttt{l} means left and \texttt{r} means right.
Finally, my original table was too large to fit a page, so I used the \textbackslash{}resizebox\{\textbackslash{}textwidth\}\{!\}\{ command.
This command needs a closing \} after the \textbackslash{}end\{tabular\} command.
This table is also now shown in the \hyperref[lst:tabs]{List of Tables} page.
\\
\textbf{Anyway, for Tables, using the LaTeX Table Generator is a great option.}

\section{Bibliography}
LaTeX{} is really convenient to deal with bibliography.
All your references should be in a \texttt{\textasteriskcentered.bib} file.
Each reference has a unique key, that you will use to refer to that publication.

You can simply cite nearly anything using the \textbackslash{}cite command.
You can cite conference papers, e.g. ``WatchIt (\cite{Perrault2013}) is an interactive wristband for smart watches.'' or journal articles, e.g. ``Lopez et al. (\cite{Lopez2017}) ran a public consultation in Mexico''.
In the first example, the key in the bib file is Perrault2013, see\\
\includegraphics{figures/bibtexkey}

\subsection{How to get Bibtex References?}
The easiest way to find the Bibtex snippet you need for a given reference is to use Google Scholar~(\cite{Scholar}).
On the main page, type the name of the paper you are looking for.
\\

In the results page, locate the paper:
\begin{figure}[!h]
  \centering
    \includegraphics[width=0.9\textwidth]{figures/scholarrefexample.png}
  \caption{Example of result on Scholar}
  \label{fig:scholarref}
\end{figure}

On the last line of the result (shown in Figure~\ref{fig:scholarref}), there is a \textbf{''} symbol.
Clicking on it will display a pop-up.
\begin{figure}[!h]
  \centering
    \includegraphics[width=0.7\textwidth]{figures/scholarpopup.png}
  \caption{Pop-up window with the possible citations}
  \label{fig:scholarpopup}
\end{figure}
\\

At the bottom (see Figure~\ref{fig:scholarpopup}), you will notice a ``Bibtex'' link. Click on it.
Scholar will then display a small block of text starting with @ symbol.
Copy and paste this snippet in your \texttt{biblio.bib} and you are done.
You may eventually want to check the citation key to something shorter.
\\

\textbf{You may get unexpected compilation errors with some references. The most common case is that the bibtex entry contains a DOI field, which in turn contains an underscore (\_).}
If that is the case, simply remove the DOI field (not a great practice but a good workaround).

\section{End of the Tips}
We are now done with the tips.
The next chapter contains more explanations and specificites of LaTeX{} and the template used here.
Good luck with your capstone report.
