%%%%%%%%%%%%%%%%%%%%%%%%%%%%%%%%%%%%%%%%%
% Masters/Doctoral Thesis
% LaTeX Template
% Version 2.5 (27/8/17)
%
% This template was downloaded from:
% http://www.LaTeXTemplates.com
%
% Version 2.x major modifications by:
% Vel (vel@latextemplates.com)
%
% This template is based on a template by:
% Steve Gunn (http://users.ecs.soton.ac.uk/srg/softwaretools/document/templates/)
% Sunil Patel (http://www.sunilpatel.co.uk/thesis-template/)
%
% Template license:
% CC BY-NC-SA 3.0 (http://creativecommons.org/licenses/by-nc-sa/3.0/)
%
%%%%%%%%%%%%%%%%%%%%%%%%%%%%%%%%%%%%%%%%%

%----------------------------------------------------------------------------------------
%	PACKAGES AND OTHER DOCUMENT CONFIGURATIONS
%----------------------------------------------------------------------------------------

% Pages (from the internal page numbering) to be printed in color - 3, 4, 5, 11, 12, 13, 14, 17, 18, 20, 21, 23, 25, 27.

\documentclass[
hidelinks,
12pt, % The default document font size, options: 10pt, 11pt, 12pt
oneside, % Two side (alternating margins) for binding by default, uncomment to switch to one side
english, % ngerman for German
doublespacing, % Single line spacing, alternatives: onehalfspacing or singlespacing
%draft, % Uncomment to enable draft mode (no pictures, no links, overfull hboxes indicated)
%nolistspacing, % If the document is onehalfspacing or doublespacing, uncomment this to set spacing in lists to single
%liststotoc, % Uncomment to add the list of figures/tables/etc to the table of contents
%toctotoc, % Uncomment to add the main table of contents to the table of contents
%parskip, % Uncomment to add space between paragraphs
%nohyperref, % Uncomment to not load the hyperref package
headsepline, % Uncomment to get a line under the header
chapterinoneline, % Uncomment to place the chapter title next to the number on one line
% consistentlayout, % Uncomment to change the layout of the declaration, abstract and acknowledgements pages to match the default layout
]{MastersDoctoralThesis} % The class file specifying the document structure

\usepackage[utf8]{inputenc} % Required for inputting international characters
\usepackage[T1]{fontenc} % Output font encoding for international characters
\usepackage{subcaption}
\usepackage{mathpazo} % Use the Palatino font by default
\usepackage{booktabs}
\usepackage{colortbl}
\usepackage[final]{pdfpages}
\usepackage{xcolor}
\usepackage{balance}
\usepackage{epigraph}
\usepackage{alltt} % for code snippet
\usepackage{listings}
\usepackage{hyperref}
\usepackage{amsmath}
\usepackage{enumitem,kantlipsum}
\usepackage{CJKutf8}
% \usepackage[backend=bibtex,style=authoryear,natbib=true]{biblatex} % Use the bibtex backend with the authoryear citation style (which resembles APA)
\usepackage[backend=bibtex,style=numeric]{biblatex}
% \usepackage[backend=bibtex,style=authoryear,natbib=true,backref=true]{biblatex} % use this line instead of the previous one if you want to use back references

\addbibresource{biblio.bib} % The filename of the bibliography


\usepackage[autostyle=true]{csquotes} % Required to generate language-dependent quotes in the bibliography

%----------------------------------------------------------------------------------------
%	MARGIN SETTINGS
%----------------------------------------------------------------------------------------

\geometry{
	paper=a4paper, % Change to letterpaper for US letter
	inner=4.0cm, % Inner margin
	outer=3.0cm, % Outer margin
	bindingoffset=.5cm, % Binding offset
	top=2.5cm, % Top margin
	bottom=2.5cm, % Bottom margin
	%showframe, % Uncomment to show how the type block is set on the page
}

%----------------------------------------------------------------------------------------
%	THESIS INFORMATION
%----------------------------------------------------------------------------------------

\thesistitle{Ghosts in the Machine: Applying \& Evaluating Explainable AI in Legal Decision Making} % Your thesis title, this is used in the title and abstract, print it elsewhere with \ttitle
\supervisor{Professor Michael \textsc{Choi}} % Your supervisor's name, this is used in the title page, print it elsewhere with \supname
\examiner{Dr/Pr. FirstName \textsc{LastName}} % Your examiner's name, this is not currently used anywhere in the template, print it elsewhere with \examname
\degree{B.Sc (Hons) and L.L.B (Hons)} % Your degree name, this is used in the title page and abstract, print it elsewhere with \degreename
\author{Tristan \textsc{Koh}} % Your name, this is used in the title page and abstract, print it elsewhere with \authorname
\addresses{} % Your address, this is not currently used anywhere in the template, print it elsewhere with \addressname

\subject{Mathematical, Computational and Statistical Sciences} % Your subject area, this is not currently used anywhere in the template, print it elsewhere with \subjectname
\keywords{Insert, keywords, here} % Keywords for your thesis, this is not currently used anywhere in the template, print it elsewhere with \keywordnames
\university{\href{https://www.yale-nus.edu.sg/}{Yale-NUS College}} % Your university's name and URL, this is used in the title page and abstract, print it elsewhere with \univname
\department{{}} % Your department's name and URL, this is used in the title page and abstract, print it elsewhere with \deptname
\group{{}} % Your research group's name and URL, this is used in the title page, print it elsewhere with \groupname
\faculty{{}} % Your faculty's name and URL, this is used in the title page and abstract, print it elsewhere with \facname

\AtBeginDocument{
\hypersetup{colorlinks=false}
\hypersetup{pdftitle=\ttitle} % Set the PDF's title to your title
\hypersetup{pdfauthor=\authorname} % Set the PDF's author to your name
\hypersetup{pdfkeywords=\keywordnames} % Set the PDF's keywords to your keywords
\hypersetup{hypertexnames=true}
}

\begin{document}
%TC:group tabular 1 1
\frontmatter % Use roman page numbering style (i, ii, iii, iv...) for the pre-content pages

\pagestyle{plain} % Default to the plain heading style until the thesis style is called for the body content

%----------------------------------------------------------------------------------------
%	TITLE PAGE
%----------------------------------------------------------------------------------------

\begin{titlepage}
% Fill out the titlepage.docx document, then save it as a pdf for inclusion here
\includepdf[pages=-,pagecommand={},width=\textwidth]{titlepage.pdf}

\end{titlepage}

%----------------------------------------------------------------------------------------
%	DECLARATION & CONSENT
%----------------------------------------------------------------------------------------

% print, sign, and scan the declaration form, then include it here
\includepdf[pages=-,pagecommand={},width=\textwidth]{declaration.pdf}

%----------------------------------------------------------------------------------------
%	ACKNOWLEDGEMENTS
%----------------------------------------------------------------------------------------
\begin{acknowledgements}
\addchaptertocentry{\acknowledgementname} % Add the acknowledgements to the table of contents
\begin{CJK*}{UTF8}{gbsn}
	他对我说: "我的恩典够你用的, 因为我的能力是在人的软弱上显得完全." 所以, 我更喜欢夸耀自己的软弱, 好使基督的能力覆庇我. - 哥林多后书 12:9
\end{CJK*}

Without God's abundant grace, this capstone and more would not have been possible. His love has been my constant source of strength in overcoming self-doubt to discover His fullness of joy.

I could not be more blessed to have my parents who have selflessly cared for me and seen me through the darkest of days, and to Law CF \& \begin{CJK*}{UTF8}{gbsn}恩岭\end{CJK*} for embracing me into the spiritual family and giving me another place to call home.

I owe a debt of gratitude to Prof. Choi \& Prof. Chesterman, who have willingly taken on this strange mixture of MCS and law, and to Prof. Danvy \& Prof. Francesca, because without their reassurance and nurture, I would never have dreamt of majoring in MCS.

I am privileged to have shared fond memories, laughter, academic and non-academic grumbling alike with Rayner, Nat, Ziying, and Shaf, who have made this five-year sojourn all the more transformative. Special mention must go to the thankless toil of the cleaning and catering aunties and uncles, without whom this whole residential experience would not have lasted more than half a day.

Finally, I thank my survey respondents because without their participation, I would only have half a capstone. To you, my reader, I hope you find this capstone as insightful to read as I did writing it.

\textit{"When the race is complete, still my lips shall repeat: Yet not I, but through Christ in me!"}
\end{acknowledgements}

%----------------------------------------------------------------------------------------
%	ABSTRACT PAGE
%----------------------------------------------------------------------------------------

\begin{abstract}
\addchaptertocentry{\abstractname} % Add the abstract to the table of contents
Natural language processing (NLP) techniques have increased substantially in performance and can potentially automate some areas of legal reasoning and writing. However, increased performance has come at the cost of increased opacity of the models. XAI has been developed to combat this opacity. Nevertheless, it is unclear how such explanations should be evaluated and whether these explanations are effective because effectiveness depends on the varying purposes and end-users of the explanations. Separately, the stakeholders of data privacy regulation (i.e. organisations, data privacy regulatory authorities, and consumers of data services) can benefit particularly from the use of XAI. Hence, this capstone fills a gap in research by evaluating the effectiveness of XAI within the context of data privacy.

This capstone trains and reports the performance of machine learning classifiers that detect data practices in PlayStore apps' data privacy policies. Then, an XAI technique, LIME, is applied on these models to produce explanations of classifications of selected data practices. Finally, the explanations are evaluated by human evaluation using a survey which uniquely utilises varied metrics and types of questions to capture the different possible dimensions of explanability. In terms of model performance, I achieve a 5 class weighted average F1 score of 66\% using support vector machine for classification with Tf-IDF. In terms of human evaluation, respondents reported a statistically significant decrease in overall trust and understandability of the models after viewing the explanations, amongst other more specific findings.

\bigskip

\noindent \textit{Keywords: eXplainable AI, human evaluation of XAI, natural language processing, machine learning, data science, data privacy}

\bigskip

\noindent Word count: 12,000 
% (original = 10,764 (incl footnotes), +50 (citations), +300 (acknowledgements) + 250 (abstract) + 250 (content page) + 500 (tables))
\end{abstract}

%----------------------------------------------------------------------------------------
% CONTRIBUTIONS
%----------------------------------------------------------------------------------------

% \chapter{Claims}
%
% This paper presents the following original contributions:
%
% \begin{enumerate}
% 	\item A hardware device for haptic sensory substitution along with designs for the construction of such a system.
% 	\item Two implementations of sensory substitution using haptic feedback, continuous and delayed feedback-based spatial navigation tasks, each of which include:
% 		\begin{enumerate}
% 			\item a front-end for providing visual input to the user during the training phase with useful readouts to the researcher,
% 			\item a transmission protocol, which maps information from the task at hand (spatial coordinates, velocity information, etc) to time-based sensor actuation signals (20$^{\circ}$ on servo 1, 35$^{\circ}$ on servo 2, etc) in real-time.
% 		\end{enumerate}
% 	\item An evaluation framework for measuring the performance of a sensory substitution system, which provides sample tasks that can be used to standardise and compare performance across the board for future research.
% 	\item A review of existing hardware and software stacks as well as possible avenues for development based on the developed metrics.
% \end{enumerate}
%
% In addition, code for displaying results in real-time, modules for managing servo overload, network latency and other factors were also written by the author.

%----------------------------------------------------------------------------------------
%	QUOTATION PAGE
%----------------------------------------------------------------------------------------
\newpage 
\vspace*{0.2\textheight}

\noindent\enquote{\itshape Any sufficiently advanced technology is indistinguishable from magic.}\bigbreak

\hfill Arthur C. Clarke

%----------------------------------------------------------------------------------------
%	LIST OF CONTENTS/FIGURES/TABLES PAGES
%----------------------------------------------------------------------------------------

\tableofcontents % Prints the main table of contents

% \listoftables % Prints the list of tables
% \label{lst:tabs}

% \listoffigures % Prints the list of figures
% \label{lst:figs}
%----------------------------------------------------------------------------------------
%	ABBREVIATIONS
%----------------------------------------------------------------------------------------

% \begin{abbreviations}{ll} % Include a list of abbreviations (a table of two columns)
%
% \textbf{LAH} & \textbf{L}ist \textbf{A}bbreviations \textbf{H}ere\\
% \textbf{WSF} & \textbf{W}hat (it) \textbf{S}tands \textbf{F}or\\
%
% \end{abbreviations}

%----------------------------------------------------------------------------------------
%	DEDICATION
%----------------------------------------------------------------------------------------

% \dedicatory{}

%----------------------------------------------------------------------------------------
%	THESIS CONTENT - CHAPTERS
%----------------------------------------------------------------------------------------

\mainmatter % Begin numeric (1,2,3...) page numbering

\pagestyle{thesis} % Return the page headers back to the "thesis" style

% Include the chapters of the thesis as separate files from the Chapters folder
% Uncomment the lines as you write the chapters

% Chapter 1

\chapter{Introduction} % Main chapter title

\label{chapter1} % For referencing the chapter elsewhere, use \ref{Chapter1}

%----------------------------------------------------------------------------------------
\section{Motivation and significance}
Natural language forms the bread and butter of the legal industry, as it is expressed in contracts, judgements and legislation. The legal industry has been adopting more machine learning tools to automate and assist low level legal analysis. Worldwide legal tech market revenues were at 27.6 billion USD and is projected to to grow at a compound annual growth rate of 4\% to 35.6 billion USD by 2027 (\cite{statista}). As early as 2018, LawGeex, a contract review startup, compared the performance of lawyers vs LawGeex's machine learning model in reviewing standard template Non-Disclosure Agreements (NDA). The model beat the humans both in terms of accuracy and time, with the model having a 94\% accuracy rate and taking 26 seconds to complete the review. In comparison, the lawyers had an average accuracy of 85\% and took 92 minutes to finish the task (\cite{lawgeex}). More significantly, at the end of 2022, an AI model GPT \footnote{The GPT which was trained by OpenAI forms the foundation of the now famous ChatGPT, which will be mentioned below.} took the US bar examination and got 50\% of the questions correct, and performed at a passing rate for both Evidence and Torts (\cite{bommarito2022}).

Like the models that were trained by LawGeex and OpenAI, most legal tech tools that aim to conduct low level legal analysis use natural language processing (NLP) techniques. NLP is a branch of AI that is concerned with giving computers the ability to understand text and spoken words in much the same way human beings can (\cite{ibm_nlp}). While NLP techniques have substantially increased in performance in recent years, it has come at the cost of explainability of their predictions because of models that are architecturally more complex (\cite{zini2022}). This lack of explainability could potentially be a significant hindrance towards their adoption within the legal industry because the lawyer / law firm which uses these models still ultimately bear the legal responsibility of ensuring that the analysis is legally sound.

Nevertheless, the intersection in skillset between data science and legal analysis is still nascent and it is unrealistic to expect all legally trained personnel to be trained in data science to the extent required to interpret the predictions of machine learning models without aid. Research within the legal NLP space have focused on building higher performing models, but there have been comparatively few papers that assess the explainability of such models. At the same time, explainable AI (XAI) techniques and research have been rising in popularity since 2020 (\cite{linardatos2020}) but have not been specifically applied onto legal text. Therefore, this project aims to bridge the gap between the lawyer and the data scientist by using Explainable AI techniques to explain the predictions of machine learning models. 

Separately, the widespread collection and use of data by organisations in recent years has led to an increase of regulations governing data privacy. This "datafication" of society includes the "transformation of interactions into data that can be valued and used for predictive analysis". Governments have therefore stepped up their efforts to guarantee privacy, with 145 countries having enacted data protection legislation in 2021 (\cite{gstrein2022}). With more sophisticated regulation comes increased difficulties for organisations to ensure that they are complying with these regulations, and for governments to enforce them. A possible area of legal tech would be tools to aid in the compliance of these regulations. Therefore, I focus on NLP and XAI in the specific context of data privacy. This context provides a realistic evaluation of the interpretability of models that are trained on legal texts relating to data privacy.

\subsection{Development and importance of explainable AI in legal technology}
With the recent release of ChatGPT (\cite{openai}) to the general public at the start of 2023, 


\subsection{Development and importance of data privacy regulation}


\section{Explanation of the APP-350 Corpus}
The APP-350 Corpus consists of 350 annotated Android app privacy policies. The corpus has been used by a previous paper to train models to detect data privacy practices (\cite{zimmeck2019}). Each annotation consists of a practice and a modality. A "privacy practice" (or "practice") describes a certain behaviour of an app that can have privacy implications (e.g., collection of a phone's device identifier or sharing of its location with ad networks). There are two modalities: \texttt{PERFORMED} (i.e. a practice is explicitly described as being performed) and \texttt{NOT\_PERFORMED} (i.e. a practice is explicitly described as not being performed).

As not all practices had modalities, altogether, 57 different categories were annotated. The following is a table of the practices and their descriptions.

\begin{table}[]
	\resizebox{\textwidth}{!}{%
	\begin{tabular}{ll}
	\hline
	Data Type                 & Description                                                                                    \\ \hline
	Contact                   & The policy describes collection of unspecified contact data.                                   \\
	Contact\_Address\_Book    & The policy describes collection of contact data from a user's address book on the phone.       \\
	Contact\_City             & The policy describes collection of the user's city.                                            \\ 
	Contact\_E\_Mail\_Address & The policy describes collection of the user's e-mail.                                          \\
	Contact\_Password         & The policy describes collection of the user's password.                                        \\
	Contact\_Phone\_Number    & The policy describes collection of the user's phone number.                                    \\
	Contact\_Postal\_Address  & The policy describes collection of the user's postal address.                                  \\
	Contact\_ZIP              & The policy describes collection of the user's ZIP code.                                        \\
	Demographic               & The policy describes collection of the user's unspecified demographic data.                    \\
	Demographic\_Age          & The policy describes collection of the user's age (including birth date and age range).        \\
	Demographic\_Gender       & The policy describes collection of the user's gender.                                          \\
	Identifier                & The policy describes collection of the user's unspecified identifiers.                         \\
	Identifier\_Ad\_ID        & The policy describes collection of the user's ad ID (such as the Google Ad ID).                \\
	Identifier\_Cookie\_or\_similar\_Tech & The policy describes collection of the user's HTTP cookies, flash cookies, pixel tags, or similar identifiers. \\
	Identifier\_Device\_ID    & The policy describes collection of the user's device ID (such as the Android ID).              \\
	Identifier\_IMEI          & The policy describes collection of the user's IMEI (International Mobile Equipment Identity).  \\
	Identifier\_IMSI          & The policy describes collection of the user's IMSI (International Mobile Subscriber Identity). \\
	Identifier\_IP\_Address   & The policy describes collection of the user's IP address.                                      \\
	Identifier\_MAC           & The policy describes collection of the user's MAC address.                                     \\
	Identifier\_Mobile\_Carrier           & The policy describes collection of the user's mobile carrier name or other mobile carrier identifier.          \\
	Identifier\_SIM\_Serial   & The policy describes collection of the user's SIM serial number.                               \\
	Identifier\_SSID\_BSSID   & The policy describes collection of the user's SSID or BSSID.                                   \\
	Location                  & The policy describes collection of the user's unspecified location data.                       \\
	Location\_Bluetooth       & The policy describes collection of the user's Bluetooth location data.                         \\
	Location\_Cell\_Tower     & The policy describes collection of the user's cell tower location data.                        \\
	Location\_GPS             & The policy describes collection of the user's GPS location data.                               \\
	Location\_IP\_Address     & The policy describes collection of the user's IP location data.                                \\
	Location\_WiFi            & The policy describes collection of the user's WiFi location data.                              \\
	SSO                       & The policy describes receiving data from an unspecified single sign on service.                \\
	Facebook\_SSO             & The policy describes receiving data from the Facebook single sign on service.                 
	\end{tabular}%
	}
	\caption{List of annotated data privacy practices and their descriptions.}
	\end{table}


The APP-350 Corpus was used in a broader project to train machine learning models to conduct a privacy census of 1,035,853 Android apps. In that project, the researchers downloaded the data privacy practices of all apps from the Play Store with more than 350 million installs (which totalled 247 apps) and 103 randomly selected apps with 5 million installs. In total, the researchers collected the data privacy policies of 350 apps.

All 350 policies were annotated by one of the authors, a lawyer with experience in data privacy law. To ensure reliability of annotations, 2 other law students were hired to double annotate 10\% of the corpus. With a mean of Krippendorff's $\alpha = 0.78$\footnote{Krippendorff's $\alpha$ is a measure of agreement, with $\alpha > 0.8$ indicating good agreement, $0.67 <= \alpha <= 0.8$ indicating fair agreement, and $\alpha < 0.67$ indicating doubtful agreement.}, the agreement between the annotations exceeded previous similar research.

For more information about how the Corpus was annotated, see the paper “MAPS: Scaling Privacy Compliance Analysis to a Million Apps”, Section 3, Pg 69 to 70. 

\subsection{Rationale for utilising the APP-350 corpus}

Since the focus of this capstone is to assess the interpretability of XAI models specifically within a legal context, this dataset was chosen for the following reasons:

\begin{enumerate}
	\item APP-350 contains real-world data privacy practices as they were scraped from Google PlayStore apps. Thus training XAI models on such a dataset would provide a realistic insight into the extent of which AI models are explainable in the legal context.
	\item Legal tech companies are also using such datasets to train models as part of their contract / document review products. By using APP-350 to train XAI models, the results can be used as a (simple)\footnote{The datasets used in industry are usually much larger and the models used are more complicated. However, APP-350 would be sufficiently complicated to serve as a toy example at an undergraduate level.} proxy for the explainability of models that are currently used in the industry.
	\item APP-350 is a labelled dataset, allowing easy validation of results. If an unlabelled dataset was used, unsupervised training would have to be conducted. The performance of the models would likely be much lower because NLP models for specific vocabulary like law are still not as sophisticated as models trained on general vocabulary. Further, there are few law specific labelled datasets to begin with. 
	\item APP-350 is labelled on both the sentence and segment (i.e. paragraph) level. This provides more granular data for training the AI models. 
\end{enumerate}


\section{Problem statement}

\section{Main findings and roadmap}


\section{Font Formatting Commands}
Similarly to Word, LaTeX provides simple formatting, including \textbf{bold}, \textit{italic}, \underline{underlined} and \texttt{ugly stuff}.
However, no underline or strikethrough by default.
You can also change the size of the text, using {\tiny tiny}, {\small small}, {\large large}, {\huge huge}.
These last commands work within a specific scope.
The scope can be specified using \{ and \}, with the \{ placed before the \textbackslash{}size command.

\subsection{Special characters}
In that case, simply use \textdollar{} (by the way, note that using the dollar sign in your text switches to mathematical notation. To actually print a dollar sign use the \textbackslash{}textdollar command).
The equation above has a label, meaning you can refer to it. The numbering system uses the chapter number (in this case 1), then the equation position within the chapter (1 again).
Example: Equation~\ref{eq:eq1} is an example of an equation in LaTeX{}.
In case you would like to have an equation without numbering it? Easy!
\begin{equation*}
t = a \times log_{2}(\frac{D}{W} + 1) + b
\end{equation*}

The only difference? The \textasteriskcentered{}  symbol in the \textbackslash{}begin\{equation\textbf{\textasteriskcentered}\}.
This also works with Figures and Tables.

\section{Code Snippets}

\begin{lstlisting}
  int main (int argc, char ** argc)
  {
    printf("Hello world!\n");
    return 0;
  }
\end{lstlisting}

This template uses the \texttt{lstlisting} package, which not the best for code snippets.
However, it works without any problem, while other packages may have compatibility issues.
Feel free to try alternative solutions, the best one being \texttt{minted}.

\section{Figures}
Figures are a bit tricky with LaTeX {\tiny(not as much as tables though)}.
Let us see a simple example below:
\begin{figure}[!h]
  \centering
    \includegraphics[width=0.9\textwidth]{figures/future.png}
  \caption{When a YNC alumni tells you that back in their days, they did not have LaTeX template and would write their report in latin on a papyrus.}
  \label{fig:future}
\end{figure}
You can refer to it: Figure~\ref{fig:future}.
This is possible thanks to the \textbackslash{}label command.
The figure should also be shown on the \hyperref[lst:figs]{List of Figures} page (note this other way of referring to another part of the manuscript!).
A common practice is use the following naming convention:
\begin{itemize}
  \item A prefix, indicating the nature of the object labelled: \texttt{eq} for equations, \texttt{fig} for figures, \texttt{tab} for tables.
  \item A colon.
  \item A unique name (easy to remember) describing your figure. Example: exp1confmatrix would suggest that the figure shows a confusion matrix for your experiment 1.
\end{itemize}

A few other points: The \textbackslash{}caption and \textbackslash{}label can be put either before or after the \textbackslash{}includegraphics command.
When you create a Figure, you need to provide placement information for LaTeX. LaTeX will usually not locate the figures \emph{exactly} where you want them.
The most common specifiers are: \texttt{h} (here), \texttt{b} (bottom of the page) and \texttt{t} (top). The \texttt{!} specifier tries to force LaTeX to put the image exactly at the location you specified (with mixed success though).
For a longer list of specifiers, please refer to: \url{https://en.wikibooks.org/wiki/LaTeX/Floats,_Figures_and_Captions}.

\subsection{Figure Size}
The size of the figure can be determined by the first parameter of the \textbackslash{}includegraphics command.
In this example, we set the size to be $0.9 \times \texttt{textwidth}$, or 90\% of the size of a column.
We could have used an absolute value in cm, e.g. \texttt{width=19cm}.

\subsection{Supported Formats}
Use standard formats, such as PNG, PDF, JPG.
LaTeX also supports other formats, such as EPS.
\textbf{Rule of thumb: use PDF as much as you can, as it uses vector graphics, making it easy to scale the figure to very large format without problems.}

\subsection{Multiple images in one figure}
You can also create complex figures with multiple images.
Here is an example, which uses a $2\times2$ layout.
The overall figure can be referred as Figure~\ref{fig:drake}.
\begin{figure}[!h]
  \begin{subfigure}[t]{.5\textwidth}
    \centering
    \includegraphics[width=\linewidth]{figures/draketl.png}
    %\caption{We could totally insert a caption here}
    %\label{fig:draketl}
  \end{subfigure}
  \hfill
  \begin{subfigure}[t]{.5\textwidth}
    \centering
    \includegraphics[width=\linewidth]{figures/draketr.png}
    %\caption{We could totally insert a caption here}
        %\label{fig:draketr}
  \end{subfigure}

  %\medskip
  % the medskip will have white space between both lines
  \begin{subfigure}[t]{.5\textwidth}
    \centering
    \includegraphics[width=\linewidth]{figures/drakebl}
    %\caption{We could totally insert a caption here}
        %\label{fig:drakebl}
  \end{subfigure}
  \hfill
  \begin{subfigure}[t]{.5\textwidth}
    \centering
    \includegraphics[width=\linewidth]{figures/drakebr}
    %\caption{We could totally insert a caption here}
    %\label{fig:drakebr}
  \end{subfigure}
  \caption{Example of a complex figures on a $2\times2$ layout.}
  \label{fig:drake}
\end{figure}

\section{Tables}
Tables can be a nightmare in LaTeX.
The easiest way to deal with tables in LaTeX is to use some online tools.
My favorite so far: \url{https://www.tablesgenerator.com/}

\textbf{Anyway, for Tables, using the LaTeX Table Generator is a great option.}


\subsection{How to get Bibtex References?}
The easiest way to find the Bibtex snippet you need for a given reference is to use Google Scholar~(\cite{Scholar}).
On the main page, type the name of the paper you are looking for.
\\

In the results page, locate the paper:
\begin{figure}[!h]
  \centering
    \includegraphics[width=0.9\textwidth]{figures/scholarrefexample.png}
  \caption{Example of result on Scholar}
  \label{fig:scholarref}
\end{figure}

On the last line of the result (shown in Figure~\ref{fig:scholarref}), there is a \textbf{''} symbol.
Clicking on it will display a pop-up.
\begin{figure}[!h]
  \centering
    \includegraphics[width=0.7\textwidth]{figures/scholarpopup.png}
  \caption{Pop-up window with the possible citations}
  \label{fig:scholarpopup}
\end{figure}
\\

At the bottom (see Figure~\ref{fig:scholarpopup}), you will notice a ``Bibtex'' link. Click on it.
Scholar will then display a small block of text starting with @ symbol.
Copy and paste this snippet in your \texttt{biblio.bib} and you are done.
You may eventually want to check the citation key to something shorter.
\\

\textbf{You may get unexpected compilation errors with some references. The most common case is that the bibtex entry contains a DOI field, which in turn contains an underscore (\_).}
If that is the case, simply remove the DOI field (not a great practice but a good workaround).

\chapter{Review of current literature} % Main chapter title


\label{chapter2} % For referencing the chapter elsewhere, use \ref{Chapter1}

%----------------------------------------------------------------------------------------
We can reference other chapters, for example, here we refer to Chapter~\ref{chapter1}.

%----------------------------------------------------------------------------------------

% Define some commands to keep the formatting separated from the content
\newcommand{\keyword}[1]{\textbf{#1}}
\newcommand{\tabhead}[1]{\textbf{#1}}
\newcommand{\code}[1]{\texttt{#1}}
\newcommand{\file}[1]{\texttt{\bfseries#1}}
\newcommand{\option}[1]{\texttt{\itshape#1}}

%----------------------------------------------------------------------------------------

%-----------------------------------
%	SECTION 1
%-----------------------------------

\section{Training models for NLP}

%-----------------------------------
%	SECTION 2
%-----------------------------------

\section{XAI methods for NLP}
Welcome to this \LaTeX{} Thesis Template, a beautiful and easy to use template for writing a thesis using the \LaTeX{} typesetting system.


%----------------------------------------------------------------------------------------

%-----------------------------------
%	SECTION 3
%-----------------------------------

\section{Evaluating the effectiveness of XAI methods}


%-----------------------------------
%	SUBSECTION 1
%-----------------------------------

\subsection{User validation of explanations}


\section{Thesis Features and Conventions}\label{ThesisConventions}

To get the best out of this template, there are a few conventions that you may want to follow.

One of the most important (and most difficult) things to keep track of in such a long document as a thesis is consistency. Using certain conventions and ways of doing things (such as using a Todo list) makes the job easier. Of course, all of these are optional and you can adopt your own method.

\subsection{References}

The \code{biblatex} package is used to format the bibliography and inserts references such as this one \parencite{Reference1}. The options used in the \file{main.tex} file mean that the in-text citations of references are formatted with the author(s) listed with the date of the publication. Multiple references are separated by semicolons (e.g. \parencite{Reference2, Reference1}) and references with more than three authors only show the first author with \emph{et al.} indicating there are more authors (e.g. \parencite{Reference3}). This is done automatically for you. To see how you use references, have a look at the \file{Chapter1.tex} source file. Many reference managers allow you to simply drag the reference into the document as you type.

The bibliography is typeset with references listed in alphabetical order by the first author's last name. This is similar to the APA referencing style. To see how \LaTeX{} typesets the bibliography, have a look at the very end of this document (or just click on the reference number links in in-text citations).

\subsection{Tables}

Tables are an important way of displaying your results, below is an example table which was generated with this code:

{\small
\begin{verbatim}
\begin{table}
\caption{The effects of treatments X and Y on the four groups studied.}
\label{tab:treatments}
\centering
\begin{tabular}{l l l}
\toprule
\tabhead{Groups} & \tabhead{Treatment X} & \tabhead{Treatment Y} \\
\midrule
1 & 0.2 & 0.8\\
2 & 0.17 & 0.7\\
3 & 0.24 & 0.75\\
4 & 0.68 & 0.3\\
\bottomrule\\
\end{tabular}
\end{table}
\end{verbatim}
}

\begin{table}
\caption{The effects of treatments X and Y on the four groups studied.}
\label{tab:treatments}
\centering
\begin{tabular}{l l l}
\toprule
\tabhead{Groups} & \tabhead{Treatment X} & \tabhead{Treatment Y} \\
\midrule
1 & 0.2 & 0.8\\
2 & 0.17 & 0.7\\
3 & 0.24 & 0.75\\
4 & 0.68 & 0.3\\
\bottomrule\\
\end{tabular}
\end{table}

You can reference tables with \verb|\ref{<label>}| where the label is defined within the table environment. See \file{Chapter1.tex} for an example of the label and citation (e.g. Table~\ref{tab:treatments}).

\subsection{Figures}

There will hopefully be many figures in your thesis (that should be placed in the \emph{Figures} folder). The way to insert figures into your thesis is to use a code template like this:
\begin{verbatim}
\begin{figure}
\centering
\includegraphics{Figures/Electron}
\decoRule
\caption[An Electron]{An electron (artist's impression).}
\label{fig:Electron}
\end{figure}
\end{verbatim}
Also look in the source file. Putting this code into the source file produces the picture of the electron that you can see in the figure below.

\begin{figure}[th]
\centering
\includegraphics{figures/electron}
\decoRule
\caption[An Electron]{An electron (artist's impression).}
\label{fig:Electron}
\end{figure}

Sometimes figures don't always appear where you write them in the source. The placement depends on how much space there is on the page for the figure. Sometimes there is not enough room to fit a figure directly where it should go (in relation to the text) and so \LaTeX{} puts it at the top of the next page. Positioning figures is the job of \LaTeX{} and so you should only worry about making them look good!

Figures usually should have captions just in case you need to refer to them (such as in Figure~\ref{fig:Electron}). The \verb|\caption| command contains two parts, the first part, inside the square brackets is the title that will appear in the \emph{List of Figures}, and so should be short. The second part in the curly brackets should contain the longer and more descriptive caption text.

The \verb|\decoRule| command is optional and simply puts an aesthetic horizontal line below the image. If you do this for one image, do it for all of them.

\LaTeX{} is capable of using images in pdf, jpg and png format.

\section{Sectioning and Subsectioning}

You should break your thesis up into nice, bite-sized sections and subsections. \LaTeX{} automatically builds a table of Contents by looking at all the \verb|\chapter{}|, \verb|\section{}|  and \verb|\subsection{}| commands you write in the source.

The Table of Contents should only list the sections to three (3) levels. A \verb|chapter{}| is level zero (0). A \verb|\section{}| is level one (1) and so a \verb|\subsection{}| is level two (2). In your thesis it is likely that you will even use a \verb|subsubsection{}|, which is level three (3). The depth to which the Table of Contents is formatted is set within \file{MastersDoctoralThesis.cls}. If you need this changed, you can do it in \file{main.tex}.

%----------------------------------------------------------------------------------------
% Chapter Template

\chapter{EDA and classifier performance} % Main chapter title

\label{chapter3} % Change X to a consecutive number; for referencing this chapter elsewhere, use \ref{ChapterX}

\section{EDA}
There are a total of 18829 annotated sentences and the top 10 frequently occurring practices make up close to 60\% of all sentences (Figure~\ref{fig:eda_data_practices}). The bottom 10 frequently occurring practices make up approximately 1\% of the dataset. There does not seem to be much variation in sentence length for the top 10 frequently occurring practices, since the standard deviation for the mean is approximately 2.5 words and the median 1.9 words (Table~\ref{tab:summary_top_10_practices}) and as seen by similar interquartile range (IQR) (Figure~\ref{fig:boxplot_top_10}). This could indicate similar sentence complexity across the practices. The mean of the sentence lengths for all top 10 practices are slightly greater than the median, which suggest some outliers with very long sentences (Figure~\ref{fig:graph_sentence_length}) that cause a slightly right skew distribution (Figure~\ref{fig:hist}). The distributions are also unimodal, with the mode around 20.

% \begin{table}[!ht]
% 	\resizebox{\textwidth}{!}{
% 	\begin{tabular}{lrrrr}
% 	\toprule
% 									  practice &  counts &  mean sentence length &  median sentence length &  percentage of total \\
% 	\midrule
% 	Identifier\_Cookie\_1stParty &    2107 &             25.4 &                    22.0 &           11.2\% \\
% 			   Contact\_E\_Mail\_Address\_1stParty &    2106 &             28.7 &                    25.0 &           11.2\% \\
% 							 Location\_1stParty &    1514 &             29.2 &                    24.0 &           8.1\% \\
% 	Identifier\_Cookie\_Tech\_3rdParty &    1250 &             27.3 &                    24.0 &           6.6\% \\
% 				Identifier\_IP\_Address\_1stParty &    1005 &             30.9 &                    27.0 &           5.3\% \\
% 				 Contact\_Phone\_Number\_1stParty &     970 &             29.1 &                    25.0 &           5.2\% \\
% 				 Identifier\_Device\_ID\_1stParty &     697 &             32.4 &                    28.0 &           3.7\% \\
% 			   Contact\_Postal\_Address\_1stParty &     597 &             28.9 &                    26.0 &           3.2\% \\
% 										   SSO &     504 &             32.6 &                    28.0 &           2.7\% \\
% 					  Demographic\_Age\_1stParty &     428 &             33.1 &                    26.0 &           2.3\% \\
% 	\bottomrule
% 	\end{tabular}
% 	}
% 	\caption{Summary statistics per practice of top 10 occurring practices.}
% 	\label{tab:top_10_sentence}
% 	\end{table}

\begin{figure}[!ht]
	\centering
	\includegraphics[width=1\textwidth]{figures/eda_data_practices.png}      
    \caption{Breakdown of data practices by proportion}
    \label{fig:eda_data_practice}
\end{figure}

\begin{table}[!ht]
	\centering
	\begin{tabular}{lrr}
		\toprule
		{} &  Mean sentence length &  Median sentence length \\
		\midrule
		count &             10.0 &               10.0 \\
		mean  &             29.8 &               25.5 \\
		std   &              2.5 &                1.9 \\
		min   &             25.4 &               22.0 \\
		25\%   &             28.7 &               24.3 \\
		50\%   &             29.1 &               25.5 \\
		75\%   &             32.0 &               26.8 \\
		max   &             33.1 &               28.0 \\
		\bottomrule
	\end{tabular}
	\caption{Summary statistics of top 10 occurring practices.}
	\label{tab:summary_top_10_practices}
\end{table}

\begin{figure}[!ht]
	\centering
	\includegraphics[width=1\textwidth]{figures/eda_mean_median.png}      
    \caption{Mean and median sentence length}
    \label{fig:graph_sentence_length}
\end{figure}


\begin{figure}[!ht]
	\begin{subfigure}[t]{.5\linewidth}
	  \centering
	  \includegraphics[width=1\linewidth]{figures/hist_eda.png}
	  \caption{Histogram of top 5 practices}
	  \label{fig:hist}
	\end{subfigure}
	\hfill
	\begin{subfigure}[t]{.5\linewidth}
		\centering
		\includegraphics[width=1\linewidth]{figures/hist_eda_2.png}
		\caption{Histogram of next top 5 practices}
		\label{fig:hist_2}
	\end{subfigure}
	\medskip
	\begin{subfigure}[b]{1\linewidth}
	  \centering
	  \includegraphics[width=0.85\linewidth]{figures/eda_boxplots.png}
	  \caption{Boxplots of sentence length (Practices shown in order of descending frequency)}
	  \label{fig:boxplot_top_10}
	\end{subfigure}
	\caption{Summary plots for top 10 frequently occurring data practices (note: not all outliers shown to focus on visualising IQR)}
	\label{fig:eda}
\end{figure}

With regard to top 10 frequently occurring 4-grams in the top 5 practices (Figure~\ref{fig:4_grams_sentence}), both 4-grams of \texttt{Cookie 1st Party} and \texttt{Cookie 3rd Party} contain many instances of "cookies", and it is hard to tell from the 4-grams alone whether they refer to collection of Cookies by 1st or 3rd parties. The same is seen to a lesser extent between the 4-grams of \texttt{Location} and \texttt{IP Address}, where both have instances of "IP address" and "mac address". In comparison, 4-grams of \texttt{Email} are clearly distinct from the rest of the data practices with unique tokens of "email" and "phone number".


\begin{figure}[!ht]
	\begin{subfigure}[t]{.5\textwidth}
	  \centering
	  \includegraphics[width=\linewidth]{figures/4_grams_cookie_1stParty.png}
	  \caption{Identifier Cookie 1st Party}
	\end{subfigure}
	\hfill
	\begin{subfigure}[t]{.5\textwidth}
	  \centering
	  \includegraphics[width=\linewidth]{figures/4_grams_cookie_3rdParty.png}
	  \caption{Identifier Cookie 3rd Party}
	\end{subfigure}
	
	\medskip
  
	\begin{subfigure}[t]{.5\textwidth}
	  \centering
	  \includegraphics[width=\linewidth]{figures/4_grams_email.png}
	  \caption{Identifier Email}
	\end{subfigure}
	\hfill
	\begin{subfigure}[t]{.5\textwidth}
	  \centering
	  \includegraphics[width=\linewidth]{figures/4_grams_ip_address.png}
	  \caption{Identifier IP Address}
	\end{subfigure}
	\begin{subfigure}[t]{.5\textwidth}
		\centering
		\includegraphics[width=\linewidth]{figures/4_grams_location.png}
		\caption{Location 1st Party}
	\end{subfigure}
	\caption{Top 10 most frequently occurring 4-grams in the top most frequently occurring 5 data practices at the sentence level}
	\label{fig:4_grams_sentence}
  \end{figure}

\section{Performance of classifiers for top N practices}
I compare the weighted precision, recall and F1 scores of the classifiers below. Since there is neither a high risk of identifying false positives or false negatives, F1 score is primarily used to assess the performance of the classifiers. As there is class imbalance in the top 10 data practices (Table~\ref{tab:top_10_sentence}), I focus on the weighted average F1 scores as the most important indicator of classifier performance. 

Given that there are in total 57 practices for the entire dataset but the top 10 data practices already comprise 60\% of all sentences, training classifiers on all 57 practices would likely lead to low performance. Thus, to find an optimal balance between classifier performance but still maintain realistic sentence complexity, I first assess the performance of the three classifiers using Tf-IDF for the top N (where $3 \le N \le 10$) frequently occurring practices. SVC performs the best across the metrics and across the top N frequently occurring practices (Figure~\ref{fig:top_n_practices}). This corresponds with the findings by Zimmeck et al. Also, performance of all the classifiers generally decreases with increasing $N$, which is not surprising since it is more difficult for the classifier to find a decision boundary with more classes. The following sections focuses on classifier performance of the top 5 practices as performance was still reasonable at 60\% weighted P/R/F1 scores. Also, since SGDClassifier performed the worst compared to the 2 other classifiers, I focus only on comparing logistic regression and SVC, with logistic regression as a baseline classifier.

\begin{figure}[!ht]
	\centering
	\includegraphics[width=1\textwidth]{figures/model_n_testing_sentence.png}      
    \caption{Weighted average P/R/F1 scores for top N ($3 \le N \le 10$) practices}
    \label{fig:top_n_practices}
\end{figure}

\section{Performance of individual classifiers for top 5 practices}
I report the individual performance of all 4 combinations of text representations and models. With regard to P/R/F1 scores (Figure~\ref{fig:heatmaps_perf}), similarly to what Zimmeck et al. found, SVC + Tf-IDF provided the best overall performance with the weighted average at 66\%. This is quite a significant performance difference from the lowest performing model, logistic regression + GloVe at 54\%. Specifically for \texttt{Cookie 1st Party}, both logistic regression + Tf-IDF and SVC + Tf-IDF performed similarly at 68\% F1, but SVC + Tf-IDF had a more balanced performance across precision and recall. With regard to the ROC curves (Figure~\ref{fig:roc_curve}), looking at the micro-averaged AUC, the classifiers all performed similarly well within a small difference of 0.03, and similarly for \texttt{Cookie 1st Party}, the difference in AUC was 0.02. GloVe embeddings seem to cause a greater variance when comparing the AUC for each class. For instance, the difference between the highest and lowest AUC for logistic regression + GloVe is $0.89 - 0.77 = 0.12$, while for logistic regression + Tf-IDF it was $0.91 - 0.85 = 0.06$.

Overall, these performance figures confirm that SVC is the better performing model, though the performance difference between the benchmark logistic regression model and SVC is not that significant (from 62\% to 66\% weighted average F1). It is also interesting to note that GloVe embeddings performed worse and caused higher variance in performance compared to Tf-IDF regardless of the model used, when GloVe is the more sophisticated text representation that can capture some semantic meaning of the tokens. These results correspond with a study measuring performance of supervised machine learning for classification tasks using generalised text datasets (\cite{hsu2020}), where LinearSVC followed by logistic regression were the highest performing models.

\begin{figure}[!ht]
	\begin{subfigure}[t]{.5\textwidth}
	  \centering
	  \includegraphics[width=\linewidth]{figures/heatmap_log_tfidf.png}
	  \caption{Logistic regression + TfIDF}
	\end{subfigure}
	\hfill
	\begin{subfigure}[t]{.5\textwidth}
	  \centering
	  \includegraphics[width=\linewidth]{figures/heatmap_log_glove.png}
	  \caption{Logistic regression + GloVe}
	\end{subfigure}
	
	\medskip
  
	\begin{subfigure}[t]{.5\textwidth}
	  \centering
	  \includegraphics[width=\linewidth]{figures/heatmap_svc_tfidf.png}
	  \caption{SVC + TfIDF}
	\end{subfigure}
	\hfill
	\begin{subfigure}[t]{.5\textwidth}
	  \centering
	  \includegraphics[width=\linewidth]{figures/heatmap_svc_glove.png}
	  \caption{SVC + GloVe}
	\end{subfigure}
	\caption{Performance heatmaps of the 4 classifiers (averaged over 5-fold cross validation)}
	\label{fig:heatmaps_perf}
  \end{figure}

\begin{figure}[!ht]
	\begin{subfigure}[t]{.5\textwidth}
	  \centering
	  \includegraphics[width=\linewidth]{figures/roc_logistic_tfidf.png}
	  \caption{Logistic regression + TfIDF}
	\end{subfigure}
	\hfill
	\begin{subfigure}[t]{.5\textwidth}
	  \centering
	  \includegraphics[width=\linewidth]{figures/roc_logistic_glove.png}
	  \caption{Logistic regression + GloVe}
	\end{subfigure}
	
	\medskip
  
	\begin{subfigure}[t]{.5\textwidth}
	  \centering
	  \includegraphics[width=\linewidth]{figures/roc_svc_tfidf.png}
	  \caption{SVC + TfIDF}
	\end{subfigure}
	\hfill
	\begin{subfigure}[t]{.5\textwidth}
	  \centering
	  \includegraphics[width=\linewidth]{figures/roc_svc_glove.png}
	  \caption{SVC + GloVe}
	\end{subfigure}
	\caption{Multiclass Receiver Operating Characteristic (ROC) Curve of the 5 data practices for each classifier}
	\label{fig:roc_curve}
  \end{figure}


% 	\begin{table}[!ht]
% 		\resizebox{\textwidth}{!}{
% 		\begin{tabular}{lrrrr}
% 		\toprule
% 								practice &  counts &  sentence\_length\_mean &  sentence\_length\_median &  counts\_percentage \\
% 		\midrule
% 		Identifier\_Mobile\_Carrier\_3rdParty &      35 &             47.1 &                    30.0 &           0.19\% \\
% 					Contact\_ZIP\_3rdParty &      34 &             40.2 &                    41.0 &           0.18\% \\
% 		Identifier\_SSID\_BSSID\_1stParty &      33 &             28.1 &                    24.0 &           0.18\% \\
% 			  Contact\_Password\_3rdParty &      33 &             24.2 &                    20.0 &           0.18\% \\
% 					Contact\_City\_3rdParty &      24 &             180 &                    14.0 &           0.13\% \\
% 		Contact\_Address\_Book\_3rdParty &      17 &             39.6 &                    34.0 &           0.1\% \\
% 			  Identifier\_IMSI\_1stParty &      13 &             48.2 &                    44.0 &           0.07\% \\
% 		Identifier\_SIM\_Serial\_3rdParty &       5 &             41.2 &                    54.0 &           0.03\% \\
% 			  Identifier\_IMSI\_3rdParty &       4 &             54.3 &                    47.5 &           0.02\% \\
% 		Identifier\_SSID\_BSSID\_3rdParty &       2 &             65.5 &                    65.5 &           0.01\% \\
% 		\bottomrule
% 	\end{tabular}
% 	}
% 	\caption{Summary statistics for bottom 10 frequently occurring practices.}
% 	\label{tab:bottom_10_sentence}
% \end{table}


% \begin{table}[!ht]
% 	\resizebox{\textwidth}{!}{
% 		\begin{tabular}{lllll}
% 		\hline
% 		\multicolumn{1}{|l|}{\textbf{Data Practice}}          & \multicolumn{1}{l|}{\textbf{Precision}} & \multicolumn{1}{l|}{\textbf{Recall}} & \multicolumn{1}{l|}{\textbf{F1}} & \multicolumn{1}{l|}{\textbf{Support}} \\ \hline
% 		\multicolumn{1}{|l|}{Contact Email Address 1st Party} & \multicolumn{1}{l|}{0.66}               & \multicolumn{1}{l|}{0.88}            & \multicolumn{1}{l|}{0.76}        & \multicolumn{1}{l|}{2106}              \\ \hline
% 		\multicolumn{1}{|l|}{Identifier Cookie 1st Party}    & \multicolumn{1}{l|}{0.59}               & \multicolumn{1}{l|}{0.78}            & \multicolumn{1}{l|}{0.68}        & \multicolumn{1}{l|}{2107}              \\ \hline
% 		\multicolumn{1}{|l|}{Identifier Cookie 3rd Party}     & \multicolumn{1}{l|}{0.64}               & \multicolumn{1}{l|}{0.35}            & \multicolumn{1}{l|}{0.45}        & \multicolumn{1}{l|}{1250}              \\ \hline
% 		\multicolumn{1}{|l|}{Identifier IP Address 1st Party} & \multicolumn{1}{l|}{0.66}               & \multicolumn{1}{l|}{0.41}            & \multicolumn{1}{l|}{0.50}        & \multicolumn{1}{l|}{1005}              \\ \hline
% 		\multicolumn{1}{|l|}{Location 1st Party}              & \multicolumn{1}{l|}{0.69}               & \multicolumn{1}{l|}{0.49}            & \multicolumn{1}{l|}{0.57}        & \multicolumn{1}{l|}{1514}              \\ \hline
% 																&                                         &                                      &                                  &                                       \\ \hline
% 		\multicolumn{1}{|l|}{\textbf{accuracy}}               & \multicolumn{1}{l|}{}                   & \multicolumn{1}{l|}{}                & \multicolumn{1}{l|}{0.64}        & \multicolumn{1}{l|}{7982}             \\ \hline
% 		\multicolumn{1}{|l|}{\textbf{macro avg}}              & \multicolumn{1}{l|}{0.65}               & \multicolumn{1}{l|}{0.58}            & \multicolumn{1}{l|}{0.59}        & \multicolumn{1}{l|}{7982}             \\ \hline
% 		\multicolumn{1}{|l|}{\textbf{weighted avg}}           & \multicolumn{1}{l|}{0.64}               & \multicolumn{1}{l|}{0.64}            & \multicolumn{1}{l|}{0.62}        & \multicolumn{1}{l|}{7982}             \\ \hline
% 		\end{tabular}
% 	}
% 	\caption{Logistic regression + Tf-IDF}
% 	\label{tab:lr+tfidf}
% \end{table}


% \begin{table}[!ht]
% 	\resizebox{\textwidth}{!}{
% 		\begin{tabular}{lllll}
% 		\hline
% 		\multicolumn{1}{|l|}{\textbf{Data Practice}}          & \multicolumn{1}{l|}{\textbf{Precision}} & \multicolumn{1}{l|}{\textbf{Recall}} & \multicolumn{1}{l|}{\textbf{F1}} & \multicolumn{1}{l|}{\textbf{Support}} \\ \hline
% 		\multicolumn{1}{|l|}{Contact Email Address 1st Party} & \multicolumn{1}{l|}{0.62}               & \multicolumn{1}{l|}{0.77}            & \multicolumn{1}{l|}{0.70}        & \multicolumn{1}{l|}{2106}              \\ \hline
% 		\multicolumn{1}{|l|}{Identifier Cookie 1st Party}    & \multicolumn{1}{l|}{0.53}               & \multicolumn{1}{l|}{0.67}            & \multicolumn{1}{l|}{0.59}        & \multicolumn{1}{l|}{2107}              \\ \hline
% 		\multicolumn{1}{|l|}{Identifier Cookie 3rd Party}     & \multicolumn{1}{l|}{0.50}               & \multicolumn{1}{l|}{0.34}            & \multicolumn{1}{l|}{0.41}        & \multicolumn{1}{l|}{1250}              \\ \hline
% 		\multicolumn{1}{|l|}{Identifier IP Address 1st Party} & \multicolumn{1}{l|}{0.47}               & \multicolumn{1}{l|}{0.22}            & \multicolumn{1}{l|}{0.30}        & \multicolumn{1}{l|}{1005}              \\ \hline
% 		\multicolumn{1}{|l|}{Location 1st Party}              & \multicolumn{1}{l|}{0.52}               & \multicolumn{1}{l|}{0.48}            & \multicolumn{1}{l|}{0.50}        & \multicolumn{1}{l|}{1514}              \\ \hline
% 																&                                         &                                      &                                  &                                       \\ \hline
% 		\multicolumn{1}{|l|}{\textbf{accuracy}}               & \multicolumn{1}{l|}{}                   & \multicolumn{1}{l|}{}                & \multicolumn{1}{l|}{0.55}        & \multicolumn{1}{l|}{7982}             \\ \hline
% 		\multicolumn{1}{|l|}{\textbf{macro avg}}              & \multicolumn{1}{l|}{0.53}               & \multicolumn{1}{l|}{0.50}            & \multicolumn{1}{l|}{0.50}        & \multicolumn{1}{l|}{7982}             \\ \hline
% 		\multicolumn{1}{|l|}{\textbf{weighted avg}}           & \multicolumn{1}{l|}{0.54}               & \multicolumn{1}{l|}{0.55}            & \multicolumn{1}{l|}{0.54}        & \multicolumn{1}{l|}{7982}             \\ \hline
% 		\end{tabular}
% 	}
% 	\caption{Logistic regression + GloVe embeddings}
% 	\label{tab:lr+glove}
% \end{table}


% \begin{table}[!ht]
% 	\resizebox{\textwidth}{!}{
% 	\begin{tabular}{lllll}
% 		\hline
% 		\multicolumn{1}{|l|}{\textbf{Data Practice}}          & \multicolumn{1}{l|}{\textbf{Precision}} & \multicolumn{1}{l|}{\textbf{Recall}} & \multicolumn{1}{l|}{\textbf{F1}} & \multicolumn{1}{l|}{\textbf{Support}} \\ \hline
% 		\multicolumn{1}{|l|}{Contact Email Address 1st Party} & \multicolumn{1}{l|}{0.74}               & \multicolumn{1}{l|}{0.81}            & \multicolumn{1}{l|}{0.77}        & \multicolumn{1}{l|}{2106}              \\ \hline
% 		\multicolumn{1}{|l|}{Identifier Cookie 1st Party}     & \multicolumn{1}{l|}{0.67}               & \multicolumn{1}{l|}{0.68}            & \multicolumn{1}{l|}{0.68}        & \multicolumn{1}{l|}{2107}              \\ \hline
% 		\multicolumn{1}{|l|}{Identifier Cookie 3rd Party}     & \multicolumn{1}{l|}{0.59}               & \multicolumn{1}{l|}{0.57}            & \multicolumn{1}{l|}{0.58}        & \multicolumn{1}{l|}{1250}              \\ \hline
% 		\multicolumn{1}{|l|}{Identifier IP Address 1st Party} & \multicolumn{1}{l|}{0.58}               & \multicolumn{1}{l|}{0.60}            & \multicolumn{1}{l|}{0.59}        & \multicolumn{1}{l|}{1005}              \\ \hline
% 		\multicolumn{1}{|l|}{Location 1st Party}              & \multicolumn{1}{l|}{0.65}               & \multicolumn{1}{l|}{0.55}            & \multicolumn{1}{l|}{0.60}        & \multicolumn{1}{l|}{1514}              \\ \hline
% 																&                                         &                                      &                                  &                                       \\ \hline
% 		\multicolumn{1}{|l|}{\textbf{accuracy}}               & \multicolumn{1}{l|}{}                   & \multicolumn{1}{l|}{}                & \multicolumn{1}{l|}{0.66}        & \multicolumn{1}{l|}{7982}             \\ \hline
% 		\multicolumn{1}{|l|}{\textbf{macro avg}}              & \multicolumn{1}{l|}{0.65}               & \multicolumn{1}{l|}{0.64}            & \multicolumn{1}{l|}{0.64}        & \multicolumn{1}{l|}{7982}             \\ \hline
% 		\multicolumn{1}{|l|}{\textbf{weighted avg}}           & \multicolumn{1}{l|}{0.66}               & \multicolumn{1}{l|}{0.66}            & \multicolumn{1}{l|}{0.66}        & \multicolumn{1}{l|}{7982}             \\ \hline
% 		\end{tabular}
% 	}
% 	\caption{SVC + Tf-IDF}
% 	\label{tab:svc+tfidf}
% \end{table}

% \begin{table}[!ht]
% 	\resizebox{\textwidth}{!}{
% 	\begin{tabular}{lllll}
% 		\hline
% 		\multicolumn{1}{|l|}{\textbf{Data Practice}}          & \multicolumn{1}{l|}{\textbf{Precision}} & \multicolumn{1}{l|}{\textbf{Recall}} & \multicolumn{1}{l|}{\textbf{F1}} & \multicolumn{1}{l|}{\textbf{Support}} \\ \hline
% 		\multicolumn{1}{|l|}{Contact Email Address 1st Party} & \multicolumn{1}{l|}{0.70}               & \multicolumn{1}{l|}{0.70}            & \multicolumn{1}{l|}{0.70}        & \multicolumn{1}{l|}{2106}              \\ \hline
% 		\multicolumn{1}{|l|}{Identifier Cookie 1st Party}     & \multicolumn{1}{l|}{0.61}               & \multicolumn{1}{l|}{0.51}            & \multicolumn{1}{l|}{0.55}        & \multicolumn{1}{l|}{2107}              \\ \hline
% 		\multicolumn{1}{|l|}{Identifier Cookie 3rd Party}     & \multicolumn{1}{l|}{0.46}               & \multicolumn{1}{l|}{0.51}            & \multicolumn{1}{l|}{0.48}        & \multicolumn{1}{l|}{1250}              \\ \hline
% 		\multicolumn{1}{|l|}{Identifier IP Address 1st Party} & \multicolumn{1}{l|}{0.37}               & \multicolumn{1}{l|}{0.49}            & \multicolumn{1}{l|}{0.42}        & \multicolumn{1}{l|}{1005}              \\ \hline
% 		\multicolumn{1}{|l|}{Location 1st Party}              & \multicolumn{1}{l|}{0.49}               & \multicolumn{1}{l|}{0.45}            & \multicolumn{1}{l|}{0.47}        & \multicolumn{1}{l|}{1514}              \\ \hline
% 																&                                         &                                      &                                  &                                       \\ \hline
% 		\multicolumn{1}{|l|}{\textbf{accuracy}}               & \multicolumn{1}{l|}{}                   & \multicolumn{1}{l|}{}                & \multicolumn{1}{l|}{0.55}        & \multicolumn{1}{l|}{7982}             \\ \hline
% 		\multicolumn{1}{|l|}{\textbf{macro avg}}              & \multicolumn{1}{l|}{0.53}               & \multicolumn{1}{l|}{0.53}            & \multicolumn{1}{l|}{0.53}        & \multicolumn{1}{l|}{7982}             \\ \hline
% 		\multicolumn{1}{|l|}{\textbf{weighted avg}}           & \multicolumn{1}{l|}{0.56}               & \multicolumn{1}{l|}{0.55}            & \multicolumn{1}{l|}{0.55}        & \multicolumn{1}{l|}{7982}             \\ \hline
% 	\end{tabular}
% 	}
% 	\caption{SVC + GloVe embeddings}
% 	\label{tab:svc+glove}
% \end{table}
% Chapter Template
\chapter{Discussion of survey results}
\label{chapter4}

In total, 31 responses were collected. The majority of respondents majored in law (58\%) vs non-law (42\%) respondents (Figure~\ref{fig:demo_1}). Except for the questions relating to subject matter expertise (data privacy and AI), the level of agreement of law vs non-law respondents about their beliefs relating to data privacy and AI were about the same (Figure~\ref{fig:demo_3}). Law respondents had less expertise in AI, while conversely, non-law respondents had less experience with data privacy. Across all respondents, while they rated that decisions by AI could be a risk to society (about 4), they also agreed that decisions by AI could be equally useful. This perhaps suggests that the respondents think the balance between "usefulness" and "risks" are not zero-sum; AI could be very helpful in solving problems, but at the same time users should be cognisant of the risks.

I mainly used sentences annotated as \texttt{Identifier Cookie 1st Party} to produce explanations for the survey. This is because the class performance of \texttt{Identifier Cookie 1st Party} performed relatively well, and has the highest number of instances in the dataset which provides a large variety of drafting styles.

\begin{figure}[!ht]
  \centering
  \includegraphics[width=0.85\linewidth]{figures/major_respondents.png}
  \caption{Breakdown of respondents' expertise by major}
  \label{fig:demo_1}
\end{figure}

% \begin{table}[!ht]
%     \resizebox{\textwidth}{!}{
%     \begin{tabular}{|p{0.45\linewidth}|l|l|}
%     \hline
%     \textbf{Major / Expected major}                    & \textbf{Count} & \textbf{Percentage of total (\%)} \\ \hline
%     Law                                                & 18             & 58                                \\ \hline
%     MCS / Computer Science / Data Science / Statistics & 3              & 9.7                               \\ \hline
%     Psychology                                         & 3              & 9.7                               \\ \hline
%     Global Affairs / Political Science                 & 2              & 6.5                               \\ \hline
%     Environmental Studies                              & 1              & 3.2                               \\ \hline
%     Economics                                          & 1              & 3.2                               \\ \hline
%     Life Sciences                                      & 1              & 3.2                               \\ \hline
%     Philosophy                                         & 1              & 3.2                               \\ \hline
%     Policy                                             & 1              & 3.2                               \\ \hline
%     \end{tabular}
%     }
%     \caption{Demographic breakdown of respondents according to academic discipline}
%     \label{tab:demo_1}
% \end{table}

\begin{figure}[!ht]
    \begin{subfigure}[b]{1\textwidth}
      \centering
      \includegraphics[width=1\linewidth]{figures/demo_3.png}
      \caption{Law vs Non-law respondents}
    \end{subfigure}
    \hfill
    \begin{subfigure}[b]{1\textwidth}
      \centering
      \includegraphics[width=1\linewidth]{figures/demo_4.png}
      \caption{All respondents}
    \end{subfigure}
    \caption{Mean scores of self-reported beliefs of respondents regarding AI \& data privacy. (1 = least agree, 5 = strongly agree. $n=31$)}
    \label{fig:demo_3}
\end{figure}

\section{Part 2 \& Part 6: Comparison of self-reported scores of explainability across the three contexts}
\label{sec:three_contexts_comparison}
Using the Wilcoxon Rank Sum Test, I tested for the following, setting $\alpha = 0.1$: 
\begin{align*}
    H0&: \text{There is no increase / decrease in scores after viewing the explainations.} \\
    H1&: \text{There is an increase / decrease in scores after viewing the explainations.}
\end{align*}
The 1-sided test was used to check whether the distribution underlying the difference between the initial and final paired scores was symmetric below or above 0 \cite{scipy}. Mathematically this difference can be stated as $d = i - f$, where $i$ and $f$ are the scores reported before and after viewing the explanations, and $d$ is the difference. Hence, if $d < 0$, then $i < f$ and the scores increased after viewing. Conversely, if $d > 0$, then $i > f$ and the scores decreased after viewing. p-values are reported in Table~\ref{tab:context_comparison} and~\ref{tab:context_comparison_2}. For Table~\ref{tab:context_comparison_2a} and~\ref{tab:context_comparison_2b}, I took the mean of self-reported scores across contexts, and across each metric respectively. For Table~\ref{tab:context_comparison_2c}, the self-reported scores were averaged across both context and metric. The trending in scores across the contexts are visualised in Figure~\ref{fig:part2_part6_comparison}.

\begin{table}[!ht]
    \centering
    \resizebox{\textwidth}{!}{    
    \begin{tabular}{|p{0.3\textwidth}|p{0.15\textwidth}|p{0.15\textwidth}|p{0.15\textwidth}|p{0.15\textwidth}|p{0.15\textwidth}|p{0.15\textwidth}|}
    \hline
        \textbf{Question} & \textbf{Context 1: Increase} & \textbf{Context 1: Decrease} & \textbf{Context 2: Increase} & \textbf{Context 2: Decrease} & \textbf{Context 3: Increase} & \textbf{Context 3: Decrease} \\ \hline
        \textbf{Do you think model is effective?} & \cellcolor{red!25}0.013 & 0.987 & 0.932 & \cellcolor{red!25}0.0684 & 0.856 & 0.144 \\ \hline
        \textbf{Do you think model is a fair method?} & 0.382 & 0.618 & 0.841 & 0.159 & 0.933 & \cellcolor{red!25}0.0671 \\ \hline
        \textbf{Do you think model is a risk to society?} & 0.756 & 0.244 & 0.428 & 0.572 & 0.825 & 0.175 \\ \hline
        \textbf{Do you trust the prediction of the model?} & 0.887 & 0.113 & 0.945 & \cellcolor{red!25}0.055 & 0.837 & 0.163 \\ \hline
    \end{tabular}
    }
    \caption{p-values comparing whether there was a statistically significant increase / decrease in the explainability scores before and after viewing explanations.}
    \label{tab:context_comparison}
\end{table}

\begin{table}[!ht]
    \centering
    \begin{subtable}[h]{0.45\textwidth}
      \centering  
      \begin{tabular}{|l|l|}
        \hline
            \textbf{Context} & \textbf{p-value} \\ \hline
            1: Increase & 0.369 \\ \hline
            1: Decrease & 0.633 \\ \hline
            2: Increase & 0.940 \\ \hline
            2: Decrease & \cellcolor{red!25}0.060 \\ \hline
            3: Increase & 0.955 \\ \hline
            3: Decrease & \cellcolor{red!25}0.0450 \\ \hline
        \end{tabular}
        \caption{p-values by metrics, taking the mean of scores across contexts}
        \label{tab:context_comparison_2a}
    \end{subtable}
    \hfill
    \begin{subtable}[h]{0.45\textwidth}
      \centering  
      \begin{tabular}{|l|l|}
            \hline  
            \textbf{Metric}                                    & \textbf{p-value}                     \\ 
            \hline
            Effective: Increase & 0.485  \\ \hline
            Effective: Decrease & 0.515  \\ \hline
            Fair: Increase      & 0.826  \\ \hline
            Fair: Decrease      & 0.174  \\ \hline
            Risk: Increase      & 0.660  \\ \hline
            Risk : Decrease     & 0.340  \\ \hline
            Trust: Increase     & 0.953  \\ \hline
            Trust: Decrease     & \cellcolor{red!25}0.0468 \\ \hline
        \end{tabular}
        \caption{p-values by contexts, taking the mean of scores across metrics}
        \label{tab:context_comparison_2b}
    \end{subtable}
    \hfill
    \begin{subtable}[h]{0.45\textwidth}
      \centering
      \begin{tabular}{|l|l|}
        \hline
        \textbf{Increase} & \textbf{Decrease} \\ \hline
        0.844             & 0.156           \\ \hline
      \end{tabular}
    \caption{p-values taking the mean scores across contexts and metrics}
    \label{tab:context_comparison_2c}
    \end{subtable}
    \caption{p-values comparing mean scores (cells highlighted in red are statistically significant)}
    \label{tab:context_comparison_2}
\end{table}

\begin{figure}[!ht]
  \centering
  \includegraphics[width=1\linewidth]{figures/part2_part6_metric_comparison.png}
  \caption{Mean scores for each metric and each context before and after respondents viewed explanations}
  \label{fig:part2_part6_comparison}
\end{figure}


Here are the instances when $p<0.1$, and therefore $H0$ can be rejected in favour of $H1$:
\begin{enumerate}
    \item Effectiveness and trust significantly decreased for context 2 (PDPC), and fairness significantly decreased for context 3 (user). Effectiveness significantly increased for context 1 (app developer) (Table~\ref{tab:context_comparison}).
    \item There is a statistically significant decrease in explainability metrics for context 1 and 2 when comparing the mean of the scores across the metrics (Table~\ref{tab:context_comparison_2a}) and for trust across the three contexts (Table~\ref{tab:context_comparison_2b}). Interestingly, there seems to be a higher correlation between the scores of the 4 metrics specifically for context 2 and 3 after the the respondents viewed the explanations (Figure~\ref{fig:part2_part6_comparison}).
\end{enumerate}

Remember that the explanations given to respondents were deliberately chosen to demonstrate the limitations of the model. Hence, respondents were likely more cognisant of such limitations\footnote{This can also be inferred from the negative trend of interpretability and understandability as explained \hyperref[sec:interpret_understand]{below}.} when they answered the same questions again in Part 6. Here are some inferences that I draw from these observations:
\begin{enumerate}
    \item \textit{"Efficiency" is dependent on the risks of making wrong predictions (App developer, PDPC)}: The speed of automation of reading privacy policies would be the same regardless whether an app developer or PDPC uses it. However, changes in efficiency differed between the 2 contexts, suggesting that automation was not the only metric that respondents considered as part of "effectiveness". Respondents could have considered that automation was more important to app developers when balanced against the risks of making a wrong prediction, as compared to the PDPC, where the risks were too high to justify automation. Hence, the classifier is not "effective" in terms of making a legally defensible decision given that it is fallible.
    \item \textit{Classifier's interpretability is especially critical for the PDPC such that trust was affected}: Respondents could have thought that PDPC, as a regulatory body in comparison to app developers and users who are not, should fully understand and explicitly justify the decision of finding an organisation (such as an app developer) liable for breaching the PDPA. Hence, when respondents reported a drop in overall interpretability and understandability, it especially affected their trust in the logic of the classifier.
    \item \textit{Fairness is perceived in relation to the user's own interests, rather than systematic fairness}: If respondents believed that the logic of the classifier was unfair because it would unfairly discriminate against a certain style of writing data privacy policies (a kind of "systemic fairness"), respondents would also have reported a drop in fairness for the PDPC context as there would be the most expectation on a regulatory body to treat data subjects and organisations equally. However, fairness only dropped for the user and not the PDPC. This could suggest that respondents viewed the drop in fairness as not being able to get recourse for their data rights which have been violated because the classifier they are relying on for evidence to the PDPC could make wrong predictions which would affect their legal claim.
    \item \textit{Expertise of the end user of explanations affects explainability}: In relation to the overall drop in explainability metrics for the contexts of the PDPC and the user, perhaps the respondents thought that PDPC and the user did not have technical knowledge to better understand the classifier's logic beyond the explanations that were presented to them, whereas the app developer would have such technical knowledge. Hence, the lack of technical expertise of the PDPC and the user caused the drop in explainability metrics.
    \item \textit{Mismatch in expectations particularly affects trust}: In the case of trust being the only metric that significantly decreased across the contexts, this could point to a mismatch of expectations. Intuitively, if respondents do not understand how a process works, they are less likely to trust the results of that process. Perhaps respondents' expectations of AI was that AI would reason similarly to how humans would, or that AI functioned entirely objectively similar to applying a formula in Excel where it is clear how the result was obtained from the formula. Instead, the explanations provided showed that the classifier was inconsistent, sometimes relying heavily on words like "cookies" and other times not so much without any discernible reason. So respondents' expectations were not met, and this caused the overall drop in trust.
\end{enumerate}

\section{Part 3: Testing whether viewing more visualisations increased explainability}
\subsection{Analysis of the reported scores of interpret and understand}
\label{sec:interpret_understand}
There is a decreasing trend of both understandability and interpretability after viewing each explanation (Figure~\ref{fig:part3_trend}, explanations can be viewed in Figure~\ref{fig:part4_explanations}). Using the Wilcoxon Rank Sum Test, I separately tested for significant differences of understandability and interpretability, setting $\alpha = 0.1$: 

\noindent H0: There is no difference in reported interpretability / understandability between the first and last questions.

\noindent H1: There is an increase / decrease of reported interpretability / understandability between the first and last questions.

\begin{table}[!ht]
  \centering
  \begin{tabular}{l|l|l|}
  \cline{2-3}
                                                & \textbf{Increase} & \textbf{Decrease} \\ \hline
  \multicolumn{1}{|l|}{\textbf{Interpretable}} & 0.867             & 0.132             \\ \hline
  \multicolumn{1}{|l|}{\textbf{Understandable}}  & 0.999             & \cellcolor{red!25}\textless{}0.001  \\ \hline
  \end{tabular}
  \caption{p-values comparing reported understandability and interpretability between the first and last question (cells highlighted in red are statistically significant)}
  \label{tab:p_values_interpret_understand}
\end{table}

\begin{figure}[!ht]
  \centering
  \includegraphics[width=1\linewidth]{figures/part3.png}
  \caption{Trend of the mean of self-reported understanding and interpretability after viewing each explanation}
  \label{fig:part3_trend}
\end{figure}

\begin{figure}[!ht]
  \begin{subfigure}[b]{1\textwidth}
    \centering
    \includegraphics[width=1\linewidth]{figures/explanations_visualisations/section_4a/Picture1.png}
    \caption{1}
  \end{subfigure}
  \hfill
  \begin{subfigure}[b]{1\textwidth}
    \centering
    \includegraphics[width=1\linewidth]{figures/explanations_visualisations/section_4a/Picture2.png}
    \caption{2}
  \end{subfigure}
  \hfill
  \begin{subfigure}[b]{1\textwidth}
    \centering
    \includegraphics[width=1\linewidth]{figures/explanations_visualisations/section_4a/Picture3.png}
    \caption{3}
  \end{subfigure}
  \hfill
  \begin{subfigure}[b]{1\textwidth}
    \centering
    \includegraphics[width=1\linewidth]{figures/explanations_visualisations/section_4a/Picture4.png}
    \caption{4}
  \end{subfigure}
  \caption{The 4 visualisations shown to respondents. Respondents were asked to rate how far they understood why the model made the prediction, and how far they found the visualisation easy to interpret. All 4 sentences were annotated as \texttt{Identifier Cookie 1st Party}.}
  \label{fig:part4_explanations}
\end{figure}

There was a statistically significant decrease in understandability but not interpretability (Table~\ref{tab:p_values_interpret_understand}). Hence, H0 can be rejected in favour of H1. 

A reason for why there was no significant decrease in interpretability was because the questions I posed to the respondents were clear in distinguishing the visualisation from the classifier's logic. Hence, respondents understood how to interpret the visualisation, but were less clear about the classifier's logic that was presented through the visualisation. 

With regards to the significant decrease in understandability, the respondents could be confused about how the model works globally after being exposed to specific predictions that were in themselves explainable, rather than being confused about a specific prediction of the model. As mentioned, the visualisations for this part were specifically chosen to demonstrate the limits of the classifier by changing keywords which were strong predictors of the data practice. This means that respondents were being exposed to a more complex (and confusing) of the model globally as they found contradictions in how the model used certain keywords in some examples but not in others. Therefore, each explanation was actually effective in communicating to the respondents how that particular prediction was made, and therefore could be considered as locally explainable. However, as a whole, respondents' understandability of the model given the contradictions seen through comparing each explanation decreased. Therefore, the classifier could be said to be locally understandable, but globally less understandable.

Another inference is that the underlying explainability method also influences the respondents' perception of the interpretability of the visualisation technique (or vice versa). This can be seen by how there is a positive correlation between the understand and interpret scores. This is not surprising since if the visualisation technique is unclear or poor, understandability of the model itself would also decrease since respondents' only way of viewing the results of the explainability technique is through the visualisation technique. To respondents, both the visualisation technique and the explainability technique are one and the same thing.

\subsection{Analysis of predicting counterfactuals}
Given the lack of any consistent trend of the respondents' votes across the questions (Figure~\ref{fig:part3_counterfactual}), it is difficult to draw any definitive interpretations from the respondents' predictions of counterfactuals. Further, the classifier itself gave ambivalent predicted probabilities for each data practice. For example (Figure~\ref{fig:part3_counterfactual_1}), the classifier predicted \texttt{Contact Email Address} at 25\% while \texttt{Cookie 1st Party} was predicted at 23\%.

\begin{figure}[!ht]
  \begin{subfigure}[b]{1\textwidth}
    \centering
    \includegraphics[width=1\linewidth]{figures/explanations_visualisations/counterfactual/3.3.png}
    \caption{Original explanation}
  \end{subfigure}
  \hfill
  \begin{subfigure}[b]{1\textwidth}
    \centering
    \includegraphics[width=1\linewidth]{figures/explanations_visualisations/counterfactual/3_3_counterfactual.png}
    \caption{Counterfactual explanation (not shown to respondents)}
    \label{fig:part3_counterfactual_1}
  \end{subfigure}
  \caption{Sample counterfactual explanation (1 out of 3)}
  \label{fig:part3_counterfactuals_example}
\end{figure}

\begin{figure}[!ht]
    \centering
    \includegraphics[width=1\linewidth]{figures/part_3_counterfactual.png}
    \caption{Votes for predicting whether counterfactual would be classified as \texttt{Identifier\_Cookie\_1st\_Party}}
    \label{fig:part3_counterfactual}
\end{figure}

\section{Part 4 \& 5: Testing which model and text representation is more explainable}
As there were three questions in each section to test the explainability of each pair of text representation and model, I totalled up the votes for each option for each part and each question (Figure~\ref{fig:part4} and~\ref{fig:part5}).

Overall, respondents found no difference in understandability between logistic regression and SVC (Figure~\ref{fig:part4}), while it was more contentious when comparing text representations, with a third split across the three options (Figure~\ref{fig:part5}). Remember that classifier performance for each pair actually did not differ significantly. In terms of comparing respondents' votes with the "ground truth" metric of model performance, most respondents would be expected to indicate that there were minimal differences between the understandability of the classifiers. This was seen when comparing logistic regression vs SVC but not seen when comparing Tf-IDF and GloVe. While there is no majority consensus for the comparison of text representations, with votes which were almost equally split instead of being heavily weighted in favour of one option show that respondents were more undecided in the understandability of the models. One inference of such results is that while respondents collectively did not give a definitive answer as to the more explainable word representation, there are still possible differences in understandability even when comparing classifiers of relatively similar performance.

\begin{figure}[!ht]
  \centering
    \begin{subfigure}[b]{0.75\textwidth}
      \includegraphics[width=1\linewidth]{figures/part4_votes.png}
      \caption{Total votes across 3 questions}
      %\label{fig:draketl}
    \end{subfigure}
    \hfill
    \centering
    \begin{subfigure}[b]{1\textwidth}
      \includegraphics[width=1\linewidth]{figures/part_4_votes_1.png}
      \caption{Votes per question}
    \end{subfigure}
    \caption{Respondents' votes to whether SVC + GloVe or Logistic regression + GloVe were more understandable}
    \label{fig:part4}
\end{figure}

\begin{figure}[!ht]
  \centering
    \begin{subfigure}[b]{0.75\textwidth}
      \includegraphics[width=1\linewidth]{figures/part5_votes.png}
      \caption{Total votes across 3 questions}
      %\label{fig:draketl}
    \end{subfigure}
    \hfill
    \centering
    \begin{subfigure}[b]{1\textwidth}
      \includegraphics[width=1\linewidth]{figures/part_5_votes_1.png}
      \caption{Votes per question}
    \end{subfigure}
    \caption{Respondents' votes to whether SVC + TfIDF or SVC + GloVe were more understandable}
    \label{fig:part5}
\end{figure}

\chapter{Conclusion} % Main chapter title
\label{chapter5} % Change X to a consecutive number; for referencing this chapter elsewhere, use \ref{ChapterX}

\section{Findings}
The research questions that were posed in Section~\ref{chap1:research_questions} can now be answered:
\begin{enumerate}
    \item \textbf{Which classifiers perform the best on a data privacy dataset in terms of traditional performance metrics?}
    
    Overall, SVC + Tf-IDF performed the best when measuring performance using weighted F1, though there was not much difference when comparing micro-averaged AUC for the ROC curves. In terms of performance when classifying \texttt{Identifier Cookie 1st Party}, both SVC + Tf-IDF and logistic regression + Tf-IDF performed similarly. It is surprising that GloVe embeddings performed worse overall and caused greater variance in the AUC of all the classes compared to Tf-IDF, given that GloVe accounts for semantic context as compared to Tf-IDF which is purely count-based.

    \item \textbf{Which classifiers are the most understandable to users that include laypersons and users with domain knowledge in data science and the law?}
    
    When SVC was compared to logistic regression, respondents by a large majority found no difference in the understandability. When Tf-IDF was compared to GloVe, respondents were undecided across the three options. Comparing the respondents' votes with performance metrics, there seems to be a slight (but non-statistical) relationship between understandability and classifier performance. Tf-IDF consistently outperformed GloVe regardless of the model used, which is reflected in how respondents were equally split between the 3 options. Whereas comparing logistic regression and SVC, the latter did not consistently outperform the former, and majority of respondents similarly found no difference between the explanations.

    \item \textbf{How would users rate the explainability of a selected XAI technique?}
    
    There was a statistically significant decrease in respondents' reported understandability but not interpretability of LIME, and there was a negative trend of both understandability and interpretability across the questions. One interpretation is that global rather than local explainability decreased because respondents were exposed to contradictory information about the model with each new explanation they viewed.

    \item \textbf{What are the differences in explainability if users were asked to consider the predictions of these classifiers from the perspective of a consumer, an organisation and the PDPC?}
    
    Effectiveness and trust significantly decreased for the PDPC context, and fairness for the consumer context. Only effectiveness significantly increased for the app developer context. Explainability metrics in general significantly decreased specifically for the PDPC and consumer context, and overall trust across all three contexts also significantly decreased. My inferences of these results were:
    \begin{enumerate}
        \item Efficiency is dependent on the risks of making wrong predictions,
        \item Classifiers' interpretability is especially critical for the PDPC such that trust was affected,
        \item Fairness is perceived in relation to the user's own interests as opposed to systemic fairness,
        \item Expertise of the end user of the explanations affects the explainability metrics; and
        \item Mismatch of expectations particularly affects trust.
    \end{enumerate}
\end{enumerate}

\subsection{Evaluation of survey design}
As mentioned in Section~\ref{chap1:research_questions}, this capstone also serves as a trial for XAI human evaluation that accounts for the different purposes of explanations. Overall, I would consider this survey design a success, given the statistically significant results reported. Nevertheless, I note a few points of improvement:

\begin{enumerate}
    \item \textit{Possible information overload:} Given the technical nature of XAI in addition to the legal context of the PDPA, a level of exposition was unavoidable as the survey was targeted at a general audience. In particular, though the three contexts could have been shorter and simpler, some level of detail was still necessary to establish the different purposes of explanation within a realistic scenario. Some respondents, especially those without any background in AI or law, later provided feedback that the survey was a challenge to complete because of the complexity of the background information needed to fully understand the contexts. Also, as I relied mostly on textual expositions, the linguistic ability of respondents could also have affected their ability to understand the background concepts. Therefore, future human evaluation should be mindful of such information overload and reliance on one medium of communication. Other forms of media could be used such as a short video primer of the background information. Alternatively, instead of showing all the contexts to a general audience, respondents of the relevant background could be shown specific contexts and explanations that are related to their particular expertise. However, the analysis of how explainability metrics change across the different contexts would then not be possible. More generally, this speaks to how technical current XAI is, such that the average person is probably unlikely to understand the visualisations without background information.
    
    \item \textit{Incorporation of qualitative evaluation:} I refrained from asking the respondents too many open-ended questions\footnote{In Part 3 of the survey, I asked respondents to explain why they provided the score they did of the understandability of the visualisation. However, due to time constraints, I did not use these qualitative responses in the analysis.} to limit the duration of the survey. However, qualitative evaluation could be used to support certain inferences made above, especially in relation to the evaluation of the changes in explainability metrics since these changes involve the interaction of different values with different purposes of the explanations. Further studies that investigate beliefs of respondents should consider supporting quantitative analyses with qualitative data.

    \item \textit{Possible interaction between model architecture and reported survey scores:} Though Part 4 \& Part 5 of the survey attempted to investigate the relationship between objective classifier performance metrics with the subjective perceptions of understandability of respondents, these performance metrics were intended to be merely a proxy for investigating whether some classifiers are inherently more understandable because of their architecture. Due to limited time and expertise, I did not consider whether there such was a relation. Therefore, further work could investigate the viability of assigning objective explainability ratings to AI models based on their inherent architecture \cite{waltl2018}, similar to car safety ratings. 
\end{enumerate}

\section{Limitations and future work}
There are a few limitations:
\begin{enumerate}
    \item \textit{Limited number of respondents ($n = 31$), diversity of expertise and age}: The original intent was to conduct a more in-depth cross-sectional study of the self-reported scores of the respondents according to their area of expertise and their beliefs relating to AI and data privacy. This analysis could be used to support certain inferences made earlier such as explainability is dependent on expertise of the end user. However, with 31 respondents, breaking down the survey results into smaller sub-groups would not reliably show statistical trends\footnote{For example, Wilcoxon Rank Sum Test is recommended to be run on populations $> 20$.}. Hence, future work could be conducted on a much larger sample size to run more cross-sectional analysis.
    
    \item \textit{Analysis is limited to the context of the APP-350 corpus, classifiers used, and LIME:} Assessing XAI is highly context dependent, and therefore the foregoing analysis should also be seen in light of these particular components. Future work to evaluate XAI within the context of data privacy could include using another data privacy dataset, other classifiers and other XAI techniques apart from LIME. In particular, LIME may interact differently with current state-of-the-art classifiers which are much more complex and rely on deep learning given that NLP has developed quite substantially since 2016 when LIME was first introduced.
    
    \item \textit{Inherent drawbacks of LIME and human evaluation of XAI:} As LIME is a post-hoc local explainability technique, respondents' understanding of the model was limited to the example visualisations that were presented to them. The sentences shown to respondents were intended to show the limitations of the classifier. This paints a more confusing and inconsistent picture of the classifier's logic, as compared to examples that consistently show "cookie" was the most important feature. In fact, this was controlled for in another study where the authors only chose to show correctly classified sentences in order to reduce the information load on the respondents \cite{gorski2021}. Hence, if more consistent examples were chosen instead, respondents might have responded differently. If post-hoc techniques are used, future work could include an interactive dashboard where respondents can choose to generate their own sentences for classification, which allows them to test their understanding of the classifier's logic is consistent.
    
    Further, human evaluations of post-hoc explanations have been criticised for being inherently flawed because of confirmation bias. Post-hoc explanations may have little to no similarities to how the underlying classifier made the prediction as they are an oversimplification of the classifier's logic. Hence, human evaluation may have limited meaning \cite{rosenfeld2021}. Instead, the authors propose four objective metrics that quantify the explanation itself and its appropriateness given the XAI goal. Future work could involve less reliance on human evaluation to incorporate more objective metrics, as well as choosing other types of XAI techniques such as global self explaining techniques. 

    \item \textit{Investigation of the differences in interaction between explainability metrics and other areas of algorithmic legal decision making:} The explainability metrics used in Section~\ref{sec:three_contexts_comparison} were presumed to be important values for a legal context because of their link to the rule of law. Nevertheless, there are other values that could be relevant such as cost because of access to justice and predictability because the legal system should provide consistent rules for its subjects to follow. Further investigation could also be conducted on how these values change and affect explainability in different areas (i.e. tort vs data privacy) and fora (i.e. PDPC vs the Court of Appeal) of the law because of different legal standards \cite{hacker2022varieties} and the expectations \& pre-conceptions of users of the legal system \cite{yalcin2022perceptions}.
    
    In particular, as the intuition behind increasing explainability is to increase trust of users, it is worth further investigating this relationship between explainability and trust. There has been evidence that shows this relation is not as general as previously thought \cite{kastner2021}. In fact, if explanations reveal problems about the system, trust could decrease rather than increase, which seems to be the case in this capstone where respondents were intentionally given examples that demonstrated the limitations of the classifiers.
\end{enumerate}

\section{Final thoughts}
This capstone has barely scratched the surface of the broader issues of XAI and the law which include defining legal standards for explainability and translating them into technical XAI requirements. To this end, the AI Act proposed by the European Commission differentiates AI systems into those which are "high risk" and "medium risk", with "high risk" systems requiring a greater level of explainability that is user-empowering and compliance-oriented \cite{sovrano2022metrics}. The Act is also likely to have greater scope than the GDPR as the Act is not limited to personal data \cite{lilian_ai_act}. Nevertheless, discussions for the Act started in April 2021 and are expected to end by late 2023. In comparison, OpenAI took less than 4 months from scoring around 45\% on the US Bar Exam using GPT-3 to coming in the 90th percentile by GPT-4 \cite{katz2023gpt}. I eagerly await the debates that follow.

%----------------------------------------------------------------------------------------
%	BIBLIOGRAPHY
%----------------------------------------------------------------------------------------

\printbibliography[heading=bibintoc]

%----------------------------------------------------------------------------------------
%	THESIS CONTENT - APPENDICES
%----------------------------------------------------------------------------------------

\appendix % Cue to tell LaTeX that the following "chapters" are Appendices

% Include the appendices of the thesis as separate files from the Appendices folder
% Uncomment the lines as you write the Appendices

%  % Appendix A

\chapter{Survey Questions} % Main appendix title

\label{AppendixA} % For referencing this appendix elsewhere, use \ref{AppendixA}

\includepdf[pages=-, pagecommand={},width=\textwidth]{../appendix/survey_questions.pdf}

% \noindent If you want to have obvious links in the PDF but not the printed text, use:

% {\small\verb!\hypersetup{colorlinks=false}!}.

%  % Appendix A

\chapter{Summary of survey results} % Main appendix title

\label{AppendixB} % For referencing this appendix elsewhere, use \ref{AppendixA}
The \texttt{.csv} containing all the responses can be found \href{https://github.com/TristanKoh/capstone-repo/blob/main/dataset/survey_results.csv}{here}.

\includepdf[pages=-,pagecommand={},width=\textwidth]{../appendix/survey_responses.pdf}

% \noindent If you want to have obvious links in the PDF but not the printed text, use:

% {\small\verb!\hypersetup{colorlinks=false}!}.


%----------------------------------------------------------------------------------------

\end{document}
